\documentclass[draft]{amsart}
\usepackage[utf8]{inputenc}
\usepackage[T1]{fontenc}
\usepackage{amssymb,amsthm}
\usepackage{mathtools}
\usepackage{tikz-cd}
\usepackage{stmaryrd} % \mapsfrom
\usepackage[shortlabels]{enumitem}
\usepackage{geometry}
\usepackage[final,kerning,spacing]{microtype}\frenchspacing
\usepackage{csquotes}
\usepackage[final]{hyperref}
\usepackage[capitalize]{cleveref}
\usepackage[inline]{showlabels}

\newcommand{\todo}[1]{\textcolor{red}{TODO: #1}}

\newcommand{\NN}{\mathbb{N}}
\newcommand{\ZZ}{\mathbb{Z}}
\newcommand{\QQ}{\mathbb{Q}}
\newcommand{\RR}{\mathbb{R}}
\newcommand{\CC}{\mathbb{C}}
\newcommand{\EE}{\mathbb{E}}
\newcommand{\FF}{\mathbb{F}}
\renewcommand{\AA}{\mathbb{A}}
\renewcommand{\SS}{\mathbb{S}}
\newcommand{\GG}{\mathbb{G}}
\newcommand{\PP}{\mathbb{P}}

\renewcommand{\H}{\mathrm{H}}
\newcommand{\D}{\mathrm{D}}
\newcommand{\R}{\mathrm{R}}
\renewcommand{\O}{\mathcal{O}}

\newcommand{\jhat}{\hat\jmath}

\newcommand{\set}[2]{\left\{#1\,\middle|\,#2\right\}}
\newcommand{\ol}[1]{\overline{#1}}
\newcommand{\ul}[1]{\underline{#1}}
\newcommand{\wh}[1]{\widehat{#1}}
\newcommand{\wt}[1]{\widetilde{#1}}
\newcommand{\cat}[1]{\mathcal{#1}}
\newcommand{\sheaf}[1]{\mathcal{#1}}

\renewcommand{\setminus}{\smallsetminus}
\renewcommand{\emptyset}{\vanothing}
\newcommand{\ssubset}{\Subset} 
\newcommand{\ssupset}{\Supset} 
\newcommand{\op}{\mathrm{op}}
\newcommand{\noloc}{:\!}

\newcommand{\too}{\longrightarrow}
\newcommand{\xto}[1]{\mathbin{\xrightarrow{#1}}}
\newcommand{\isoto}{\mathbin{\xrightarrow{\sim}}}
\newcommand{\injto}{\mathbin{\hookrightarrow}}
\newcommand{\injtoo}{\mathbin{\longhookrightarrow}}
\newcommand{\epito}{\mathbin{\to\kern-.8em\to}}
\newcommand{\epitoo}{\mathbin{\too\kern-.8em\to}}

\renewcommand{\Pr}{\mathrm{Pr}}
\newcommand{\Stab}{\mathrm{Stab}}

\newcommand{\one}{\mathbf{1}}
\newcommand{\blank}{-} % todo

\DeclareMathOperator{\Spec}{Spec}
\DeclareMathOperator{\Hom}{Hom}
\DeclareMathOperator{\Anima}{Ani}
\DeclareMathOperator{\LCHaus}{LCHaus}
\DeclareMathOperator{\fib}{fib}
\DeclareMathOperator{\id}{id}
\DeclareMathOperator{\coShv}{coShv}
\DeclareMathOperator{\End}{End}
\DeclareMathOperator{\Mot}{Mot}
\DeclareMathOperator{\CAlg}{CAlg}
\DeclareMathOperator{\Spa}{Spa}
\DeclareMathOperator{\dlog}{dlog}
\DeclareMathOperator{\SingularSupport}{SS}
\DeclareMathOperator{\Ind}{Ind}
\DeclareMathOperator{\Map}{Map}
\DeclareMathOperator{\Open}{Open}
\DeclareMathOperator{\Seminorm}{Seminorm}
\DeclareMathOperator{\map}{map}
\DeclareMathOperator{\Shv}{Shv}
\DeclareMathOperator{\Nuc}{Nuc}
\DeclareMathOperator{\Cat}{Cat}
\DeclareMathOperator{\CompAss}{CompAss}
\DeclareMathOperator{\Sp}{Sp}
\DeclareMathOperator{\Fun}{Fun}
\DeclareMathOperator{\Mod}{Mod}
\DeclareMathOperator{\Perf}{Perf}
\DeclareMathOperator{\CompHaus}{CompHaus}
\DeclareMathOperator{\Spf}{Spf}
\DeclareMathOperator{\iHom}{\ul{Hom}}
\DeclareMathOperator{\PSh}{PSh}
\DeclareMathOperator{\Corr}{Corr}
\DeclareMathOperator{\Pro}{Pro}
\DeclareMathOperator{\Image}{Im}
\DeclareMathOperator{\Loc}{Loc}
\DeclareMathOperator{\Tate}{Tate}
\DeclareMathOperator{\Ext}{Ext}
\DeclareMathOperator{\Vect}{Vect}
\DeclareMathOperator{\Spv}{Spv}
\DeclareMathOperator{\Calk}{Calk}
\DeclareMathOperator{\QCoh}{QCoh}
\DeclareMathOperator{\GL}{GL}
\DeclareMathOperator*{\indinjlim}{\text{``}\varinjlim\text{''}} % todo
\DeclareMathOperator*{\indprojlim}{\text{``}\varprojlim\text{''}} % todo
\DeclareMathOperator*{\colim}{colim}

\newtheorem{thm}{Theorem}[section]
\newtheorem{prop}[thm]{Proposition}
\newtheorem{lem}[thm]{Lemma}
\newtheorem{cor}[thm]{Corollary}

\theoremstyle{definition}
\newtheorem{defn}[thm]{Definition}
\newtheorem{rem}[thm]{Remark}
\newtheorem{properties}[thm]{Properties}
\newtheorem{fact}[thm]{Fact}
\newtheorem{ex}[thm]{Example}
\newtheorem{exercise}[thm]{Exercise}
\newtheorem{corrections}[thm]{Corrections}
\newtheorem{question}[thm]{Question}
\newtheorem{warning}[thm]{Warning}
\newtheorem*{claim}{Claim}
\newtheorem*{notation}{Notation}

\begin{document}
\title{Continuous K-Theory and Geometric Topology}
\author{Thomas Nikolaus}
\maketitle
\tableofcontents

\section{Talk 1}
Recall that $\Pr^L_{\mathrm{st}}$ is the category of presentable stable categories. We denote by $\otimes$ the Lurie tensor product; then $\Fun^L(\cat C, \cat D)$ is the internal Hom and the category $\Sp$ of spectra is the unit for $\otimes$.

\begin{defn}
$\cat C$ is \emph{dualizable} if $\cat C$ is dualizable in $(\Pr^L_{\mathrm{st}}, \otimes)$. The dual is then given by $\cat C^\vee = \Fun^L(\cat C, \Sp)$.
\end{defn}

\begin{defn}
\begin{enumerate}[(i)]
\item A functor $F\colon \cat D\to \cat E$ is called a \emph{homological epi} if the restriction $\Ind(\cat E)\to \Ind(\cat D)$ is fully faithful or, equivalently, if the induced functor $\Ind(\cat D)\to \Ind(\cat E)$ is a Bousfield localization.

\item A map $R\to S$ of ring spectra is a \emph{homological epi} if the following equivalent conditions hold:
\begin{itemize}
\item the induced functor $\Mod(R)^{\omega} \to \Mod(S)^\omega$ is a homological epi.
\item $\Mod(R) \to \Mod(S)$ is a Bousfield localization.
\item the restriction $\Mod(S) \to \Mod(R)$ is fully faithful.
\item $S\otimes_RS \isoto S$.
\item $S\sqcup_RS \isoto S$, where the pushout is taken in $\EE_1$-rings.
\item $I\otimes_RI \isoto I$, where $I = \fib(R\to S)$.
\end{itemize}

\item A non-unital ring spectrum $R$ is called \emph{H-unital} if $R^+ = R\oplus \SS \to \SS$ is a homological epi, where $\SS$ is the sphere spectrum.
\end{enumerate}
\end{defn}

\begin{defn}
\begin{enumerate}[(i)]
\item A map $x\to y$ in $\cat C$ is called \emph{(weakly) compact} if for every filtered colimit $d = \colim_i d_i$ and every map $y\to d$, the composite $x\to y\to d$ factors over some $d_{i_0}$:
\[
\begin{tikzcd}
x \ar[d,dashed] \ar[r] & y \ar[d] \\
d_{i_0} \ar[r] & \colim_i d_i.
\end{tikzcd}
\]

\item An object $x \in \cat C$ is called \emph{compactly exhaustible} if it can be written as
\[
x = \colim (x_0\to x_1\to \dotsb)
\]
with compact transition maps $x_i\to x_{i+1}$.

\item An object $x\in \cat C$ is called \emph{transfinitely compactly exhaustible} if it can be written as
\[
x = \colim_{i\in I} x_i,
\]
where $I$ is filtered, without terminal object, and \emph{antisymmetric} (i.e., the non-invertible morphisms in $I$ form an ideal; equivalently one-sided inverses are invertible), such that for every non-invertible map $i\to j$ in $I$, the induced map $x_i\to x_j$ is compact.
\end{enumerate}
\end{defn}

\begin{thm}\label{Nikolaus:dualizable}
For $\cat C \in \Pr^L_{\mathrm{st}}$ the following are equivalent:
\begin{enumerate}[(1)]
\item $\cat C$ is dualizable.

\item $\cat C$ is a retract in $\Pr^L_{\mathrm{st}}$ of a compactly generated stable category, i.e., $\cat C = \Ind(\cat C_0)$ for a small stable category $\cat C_0$.

\item $\cat C$ is the kernel of 
\[
\Ind(\cat D) \xrightarrow{\Ind(F)} \Ind(\cat E),
\]
where $\cat D, \cat E$ are small stable categories and $F\colon \cat D\to \cat E$ is an exact homological epi.

\item $\cat C$ is the kernel of 
\[
\Mod(R) \to \Mod(S)
\]
for a map $R\to S$ of ring spectra which is a homological epi. In this case we write $\cat C = \Mod(R,I)$.

\item $\cat C$ is the kernel of 
\[
\Mod(R^+) \to \Mod(\SS) = \Sp,
\]
where $R$ is an H-unital ring spectrum. In this case we write $\cat C = \Mod_\H(R)$.

\item The colimit functor $k\colon \Ind(\cat C) \to \cat C$ admits a left adjoint $\jhat$ (which is automatically fully faithful since the right adjoint $j$ of $k$ is fully faithful).

\item $\cat C$ is $\omega_1$-compactly generated and the colimit
\[
k\colon \Ind(\cat C^\omega_1) \to \cat C
\]
admits a left adjoint.

\item $\cat C$ is generated under colimits by compactly exhaustible objects.

\item Every object in $\cat C$ is transfinitely compactly exhaustible.

\item (AB6) Products in $\cat C$ distribute over filtered colimits: the natural map
\[
\colim_{(i_k)_k \in \prod_{k\in K}I_k} \prod_{k\in K} x_{k,i_k} \isoto
\prod_{k\in K} \colim_{i\in I_k} x_{k,i}
\]
is an isomorphism in $\cat C$.\footnote{Since we are in a stable category, we may equivalently replace products with limits. Moreover, one can assume that all $I_k$'s are the same.}
\end{enumerate}
\end{thm}
\begin{proof}[\enquote{Proofs}:]
If $\cat C$ is dualizable, we have a Bousfield localization $\Ind(\cat C^\kappa)\to \cat C$. We then obtain a lifting diagram
\[
\begin{tikzcd}
\Ind(\cat C^\kappa) \otimes \cat C^\vee \ar[r] & \cat C\otimes \cat C^\vee \\
& \Sp \ar[u,"\text{coev}"'] \ar[ul,dashed],
\end{tikzcd}
\]
where the lift exists, because a functor from $\Sp$ is uniquely determined by specifying the image of $\SS$. Reinterpreting, we obtain a lift
\[
\begin{tikzcd}
& \Ind(\cat C^\kappa) \ar[d] \\
\cat C \ar[ur,dashed] \ar[r,equals] & \cat C.
\end{tikzcd}
\]

For the implication (2)$\implies$(6), write $\cat C = \Ind(\cat C_0)$. Then we have
\[
\jhat = \Ind(\cat C_0\injto \cat C) \colon \cat C = \Ind(\cat C_0)\to \Ind(\cat C).
\]
To pass to retracts, use some abstract argument.

For the implication (6)$\implies$(2), use $\jhat\colon \cat C \to \Ind(\cat C)$.

Ad (7)$\implies$(3): Consider 
\[
\cat C \xrightarrow{\jhat} \Ind(\cat C^{\omega_1}) \to \Ind(\cat C^{\omega_1})/\cat C = \Ind(\Calk^{\mathrm{cont}}(\cat C)).
\]
To see the equality, use that $\cat C^{\omega_1} \xrightarrow{\mathrm{pr}} \Calk^{\mathrm{cont}}(\cat C)$ is a homological epi.

Finally, for (6)$\iff$(10), note that distributivity is equivalent to $k\colon \Ind(\cat C)\to \cat C$ preserving products. (This relies on the fact that the category of anima satisfies (AB6).)
\end{proof}

\begin{ex}
The category $\Shv(X)$ of sheaves with values in $\Sp$ on a locally compact space $X$ is generated by sheaves of the form $\Sigma^\infty_+ \ul{U}$, where $\ul U$ is the sheaf on $X$ represented by $U$ and $\Sigma^\infty_+\colon \Anima \to \Sp$ is the suspension functor. Every inclusion $U \injto V$ in $\Open(X)$ factors through a compact subset $K$, that is, $U \to K\to V$. It follows that $\Sigma^\infty_+ \ul{U} \to \Sigma^\infty_+\ul{V}$ is a compact map. Finally, $\Sigma^\infty_+\ul{U}$ is compactly exhaustible if $U$ is, and every $U$ is a filtered colimit of such. Therefore, $\Shv(X)$ is dualizable.
\end{ex}

\begin{properties}[of dualizable categories]
Let $\cat C, \cat D$ be dualizable categories.
\begin{enumerate}[(a)]
\item An object $x\in \cat C$ is compactly exhaustible if and only if $x$ is $\omega_1$-compact.

\item A functor $F\colon \cat C\to \cat D$ in $\Pr^L$ is strongly continuous (meaning that $F$ admits a right adjoint that preserves colimits) if and only if $F$ preserves compact morphisms, if and only if the following diagram commutes:
\[
\begin{tikzcd}
\Ind(\cat C) \ar[r,"\Ind(F)"] & \Ind(\cat D) \\
\cat C \ar[u,"\jhat"] \ar[r,"F"'] & \cat D \ar[ul,Rightarrow,shorten=4mm, "\sim"'] \ar[u,"\jhat"'].
\end{tikzcd}
\]

\item A morphism $x\to y$ in $\cat C$ is compact if it lifts to a map $jx \to \jhat y$ in $\Ind(\cat C)$. In this case, we define the space of \emph{compactly assembled maps} as the spectrum
\[
\map_{\cat C}^{\mathrm{ca}}(x,y)\coloneqq \map_{\Ind(\cat C)}(jx, \jhat y).
\]

\item If $x = \colim_{i\in I} x_i$ is $I$-compactly exhaustible, then $\jhat(x) = \colim_{i\in I} jx_i$ in $\Ind(\cat C)$.

\item Recall the resolution 
\[
\cat C \to \Ind(\cat C^{\omega_1}) \to \Ind(\Calk^{\mathrm{cont}}(\cat C)),
\]
and denote $p\colon \cat C^{\omega_1} \to \Calk^{\mathrm{cont}}(\cat C)$ the projection. Then
\[
\map_{\Calk(\cat C)}(px, py) = \map_{\cat C}(x,y) / \map_{\cat C}^{\mathrm{ca}} (x,y).
\]
\end{enumerate}
\end{properties}

\begin{defn}
We denote $\Pr^L_{\mathrm{dual}}$ the category of dualizable categories with strongly left adjoint functors.
\end{defn}

\begin{cor}
A left adjoint functor $\Shv(X) \to \cat D$ is strongly left adjoint if the corresponding cosheaf $\sheaf F\colon \Open(X) \to \cat D$ satisfies the following condition: for every $U\ssubset V$, the induced map $\sheaf F(U) \to \sheaf F(V)$ is a compact morphism in $\cat D$.
\end{cor}

\begin{defn}
We define the continuous K-theory of a dualizable category $\cat C$ as the fiber
\[
K^{\mathrm{cont}}(\cat C) = \fib\bigl(K(\cat C^{\omega_1}) \to K(\Calk^{\mathrm{cont}}(\cat C))\bigr) \cong \Omega K(\Calk^{\mathrm{cont}}(\cat C)).
\]
\end{defn}


\section{Talk 2: Verdier duality and 6 functors}
\subsection{Recap}
Let $\cat C$ be a dualizable category. Recall that we have a resolution
\[
\cat C \to \Ind(\cat C^{\omega_1}) \to \Ind\Calk^{\mathrm{cont}}(\cat C).
\]
The category $\Pr^L_{\mathrm{dual}}$ of dualizable categories with strongly continuous functors is itself a presentable category (due to Ramzi), and the forgetful functor $\Pr^L_{\mathrm{dual}} \to \Pr^L$ preserves colimits. 

\begin{exercise}
To compute colimits in $\Pr^L$, use $\Pr^L \simeq (\Pr^R)^{\op}$ and use that $\Pr^R \to \Cat$ commutes with limits. Use this to prove that $\Pr^L_{\mathrm{dual}} \to \Pr^L$ commutes with colimits.
\end{exercise}

\begin{defn}
Let $\cat C$ be a dualizable category. We define the \emph{continuous K-theory} as
\[
K^{\mathrm{cont}}(\cat C)\coloneqq \fib\bigl(K(\cat C^{\omega_1}) \to K(\Calk^{\mathrm{cont}}(\cat C))\bigr) \simeq \Omega K(\Calk^{\mathrm{cont}}(\cat C)).
\]
\end{defn}

\begin{rem}
In increasing generality, K-theory has been defined in the following setups:
\begin{enumerate}[(i)]
\item for rings $R$;
\item for additive categories (e.g., $\mathrm{Proj}_R$);
\item for (small) stable categories (e.g., $\Mod(R)^{\omega}$);
\item for dualizable categories (e.g., $\Mod(R)$ or $\Mod(R,I)$).
\end{enumerate}
\end{rem}

\subsection{Verdier duality and 6 functors}
Let $f\colon Y\to X$ be a continuous map. Then we have an adjunction
\[
\begin{tikzcd}
f^* \colon \Shv(X)\ar[r,shift left] & \ar[l,shift left] \Shv(Y) \noloc f_*.
\end{tikzcd}
\]
If $f$ is locally separated and locally proper (due to Schn\"urer and Soergel), we have another adjunction
\[
\begin{tikzcd}
f_! \colon \Shv(Y)\ar[r,shift left] & \ar[l,shift left] \Shv(X) \noloc f^!.
\end{tikzcd}
\]
Moreover, on $\Shv(X)$ we have a symmetric monoidal structure $\otimes$ and an internal hom $\iHom$.

\begin{defn}
A commutative algebra $\cat C \in \CAlg(\Pr^L_{\mathrm{st}})$ is called \emph{locally rigid} if:
\begin{enumerate}[(i)]
\item The functor $\otimes\colon \cat C\otimes \cat C \to \cat C$ admits a cocontinuous right adjoint $\Delta\colon \cat C \to \cat C\otimes \cat C$ which is a $\cat C$-$\cat C$-bimodule map.\footnote{Note that this is just a \emph{condition} and not additional structure, since $\Delta$ is automatically a lax bimodule map.}

\item $\cat C$ is dualizable; equivalently there exists a counit $\cat C\to \Sp$ for the comultiplication $\Delta$.
\end{enumerate}

We call $\cat C$ \emph{rigid} if in addition $\one \in \cat C^\omega$.
\end{defn}

\begin{ex}
\begin{enumerate}[(a)]
\item Let $R$ be a commutative ring. Then $\Mod(R)$ is rigid, because we have
\[
\begin{tikzcd}
\Mod(R) \otimes \Mod(R) \ar[r,"\otimes_R"] \ar[d,"\simeq"'] & \Mod(R) \ar[dl,shift left, hook, "\mathrm{res}"] \\
\Mod(R\otimes_{\SS}R ) \ar[ur,shift left],
\end{tikzcd}
\]
where the upper diagonal map is given by base-change along $m\colon R\otimes_{\SS} R \to R$. 

\item For a homological epi $R\to R/I$ of commutative rings, the category $\Mod(R,I)$ is locally rigid, and it is rigid if and only if $I$ is compact as an $R$-module.

\item A small stable category $\cat C$ is rigid if and only if $\Ind(\cat C)$ is rigid.

\item Let $X \in \LCHaus$. Then $\Shv(X)$ is locally rigid; if moreover $X$ is compact, then $\Shv(X)$ is rigid. To see this, note that we have a commutative diagram
\[
\begin{tikzcd}
\Shv(X) \otimes \Shv(X) \ar[dr,bend left, "\otimes"] \ar[d,"\simeq"'] \\
\Shv(X\times X)\ar[r,shift left,"\Delta^*"] & \ar[l,shift left,"\Delta_*"] \Shv(X)
\end{tikzcd}
\]
and note that $\Delta_* = \Delta_!$, so that $\Delta_*$ has a right adjoint. The Frobenius identity (that is, the fact that $\Delta_*$ is a bimodule map) follows from the projection formula.

The counit is given by $\Gamma_c = p_!\colon \Shv(X) \to \Sp$, where $p\colon X\to \mathrm{pt}$ is the tautological map.

\item The category $D(\ZZ)^\wedge_p$ is compactly generated and locally rigid. But it is not rigid, because the unit $\ZZ$ is not compact.
\end{enumerate}
\end{ex}

\begin{prop}
\begin{enumerate}[(1)]
\item If $\cat C$ is locally rigid, then
\[
\Sp \xrightarrow{\mathrm{unit}} \cat C \xrightarrow{\Delta} \cat C\otimes \cat C
\]
exhibits $\cat C$ as a Frobenius algebra (i.e., the composition is the coevaluation for a self-duality on $\cat C$). In particular, $\cat C \simeq \cat C^\vee$.

\item The counit $\cat C\to \Sp$ (which is dual to the unit $\Sp \to \cat C$) is equivalent to 
\begin{align*}
\cat C &\to \Sp, \\
X &\mapsto \map^{\mathrm{ca}}(\one,\blank) \simeq \Gamma_c(\blank).
\end{align*}
The self-duality is exhibited by the equivalence
\begin{align*}
\cat C &\isoto \Fun^L(\cat C, \Sp), \\
X &\mapsto \map^{\mathrm{ca}}(\one, X\otimes\blank), \\
(F\otimes \id)(\Delta(\one)) &\mapsfrom F.
\end{align*}
\end{enumerate}
\end{prop}

\begin{ex}
We have 
\[
\Shv(X) \simeq \Shv(X)^\vee \simeq \coShv(X),
\]
which is known as Verdier duality. The evaluation map for this duality is given by
\begin{align*}
\Shv(X) \otimes \Shv(X) &\to \Sp, \\
(\sheaf F, \sheaf G) &\mapsto \Gamma_c(\sheaf F\otimes \sheaf G).
\end{align*}
For a map $f\colon Y\to X$ in $\LCHaus$, the functor $f_!$ is dual to $f^*$.
\end{ex}

\begin{prop}
Let $\cat C$ be locally rigid and $\cat M$ a $\cat C$-module (in $\Pr^L_{\mathrm{st}}$). Then $\cat M$ is dualizable relative to $\cat C$ (i.e., dualizable in $\Mod_{\cat C}(\Pr^L_{\mathrm{st}})$) if and only if $\cat M$ is dualizable in $\Pr^L_{\mathrm{st}}$.
\end{prop}

\begin{ex}
A $\ZZ$-linear stable category is dualizable over $\ZZ$ if and only if it is dualizable over $\SS$.
\end{ex}

\begin{defn}
A morphism $f\colon x\to y$ in a closed symmetric monoidal category $\cat C$ is called \emph{trace class} if it lifts as follows:
\[
\begin{tikzcd}
& \iHom(x,\one) \otimes y \ar[d] \\
\one \ar[ur,dashed,"\exists"] \ar[r,"f"'] & \iHom(x,y).
\end{tikzcd}
\]
\end{defn}

\begin{thm}[Clausen, Ramzi, Scholze]
Let $\cat C \in \CAlg(\Pr^L_{\mathrm{st}})$, such that the underlying category is dualizable, then:
\begin{enumerate}[(a)]
\item $\cat C$ is locally rigid if and only if 
\[
\{\text{compact morphisms}\} \subseteq \{\text{trace class morphisms}\}.
\]
\item The unit $\one\in \cat C$ is compact if and only if 
\[
\{\text{trace class morphisms}\} \subseteq \{\text{compact morphisms}\}.
\]
\item $\cat C$ is rigid if and only if the classes of compact morphisms and of trace class morphisms agree.
\end{enumerate}
\end{thm}

\begin{ex}
In order to see that $\Shv(X)$ is locally rigid, it thus suffices to see that the maps $\Sigma^\infty_+\ul U \to \Sigma^\infty_+\ul V$ are trace class for all $U \ssubset V$.
\end{ex}

\begin{defn}
Let $\cat A \to \cat B$ be a map in $\CAlg(\Pr^L_{\mathrm{st}})$. Then $\cat B$ is called \emph{locally rigid over $\cat A$} if:
\begin{enumerate}[(i)]
\item The multiplication $\cat B\otimes_{\cat A}B \to \cat B$ has an $\cat A$-linear and cocontinuous right adjoint $\Delta \colon \cat B\to \cat B\otimes_{\cat A} \cat B$ which is a $\cat B$-$\cat B$-bimodule map.

\item $\cat B$ is dualizable relative to $\cat A$ (i.e., $\cat B$ is dualizable in $\Mod_{\cat A}(\Pr^L_{\mathrm{st}})$); equivalently, there exists a counit $\cat B\to \cat A$ for the comultiplication $\Delta$.
\end{enumerate}
\end{defn}

\begin{ex}
Let $f\colon Y\to X$ be a locally proper and separated map of topological spaces. Then $\Shv(Y)$ is locally rigid over $\Shv(X)$.
\end{ex}

\begin{prop}
Let $\cat A\to \cat B$ be locally rigid. 
\begin{enumerate}[(a)]
\item A $\cat B$-module $\cat M$ is dualizable over $\cat B$ if and only if $\cat M$ is dualizable over $\cat A$.

\item Given an algebra map $\cat B\to \cat C$, then $\cat C$ is locally rigid relative to $\cat B$ if and only if $\cat C$ is locally rigid relative to $\cat A$.
\end{enumerate}
\end{prop}

\begin{thm}
The category $\CAlg(\Pr^L_{\mathrm{st}})^\op$ carries a 6-functor formalism in which the exceptional maps are locally rigid maps.
\end{thm}

\section{Talk 3}

\begin{defn}
A \emph{simple anima} is a compact anima $Z$ together with a lift
\[
\begin{tikzcd}
\chi^{\mathrm{loc}} \ar[r,phantom, "\in"] &[-2em] A(\mathrm{pt})\otimes Z \ar[r,"\text{Assembly}"] 
& A(Z)\ar[d,"\simeq"] &[-1.5em] \ar[l,phantom, "\ni"] \chi = {[\SS]} \\
& & K((\Sp^Z)^\omega) \ar[r,equals] & K(\SS[\Omega Z])
\end{tikzcd}
\]
The $\infty$-groupoid $\Anima^{\mathrm{simple}}$ is defined as the pullback
\[
\begin{tikzcd}
\Anima^{\mathrm{simple}} \ar[d] \ar[r] \ar[dr,phantom, very near start, "\lrcorner"] & \Anima_{*/} \ar[d] \\
(\Anima^{\omega})^{\simeq} \ar[r] & \Anima,
\end{tikzcd}
\]
where the bottom map is given by mapping a compact anima $\chi$ to the anima of lifts $\chi^{\mathrm{loc}}$ of $\chi$.
\end{defn}

\begin{thm}[Wall '65]
A compact anima is a finite anima if and only if it refines to a simple anima.
\end{thm}

\begin{thm}[Whitehead '50]
A homotopy equivalence between finite CW complexes is simple (i.e., homotopic to a composition of elementary expansion and collapse maps) if and only if it refines to a map in $\Anima^{\mathrm{simple}}$.
\end{thm}

\begin{thm}[Hatcher, Waldhausen, Waldhausen--Jahren--Rognes]
The $\infty$-groupoid $\Anima^{\mathrm{simple}}$ is equivalent to Hatcher's classifying space of simple homotopy types, i.e., the geometric realization
\[
\lvert \{\mathrm{Polyhedra}, \textnormal{simple maps}\}\rvert \simeq \lvert \{ \mathrm{sSet}^{\mathrm{fin}}_{\mathrm{nd}}, \textnormal{simple maps}\}\rvert.
\]
\end{thm}

\begin{thm}
\begin{enumerate}[(1)]
\item West '77: Every compact manifold (AMR) has the homotopy type of a finite CW complex.

\item Chapman '65: Every homeomorphism between finite CW complexes is a simple homotopy equivalence.
\end{enumerate}
\end{thm}

We construct a functor
\[
\begin{Bmatrix}
\text{nice compact topological} \\
\text{spaces with homeomorphisms}
\end{Bmatrix} \to \Anima^{\mathrm{simple}}.
\]

Recall: Let $X$ be a locally compact Hausdorff space. We have seen that $\Shv(X;\Sp)$ is dualizable.

\begin{defn}
Let $\cat C$ be dualizable. We define
\[
\wh{\coShv}(X;\cat C) \coloneqq \iHom^{\mathrm{dual}}(\Shv(X);\cat C) \subseteq \Ind(\coShv(X;\cat C)).
\]
It is covariantly functorial in proper maps $f\colon X\to Y$ induced by $f^*\colon \Shv(Y) \to \Shv(X)$.

Moreover, we define 
\[
\wh{\coShv}_{\mathrm{cs}}(X;\cat C) = \colim_{\substack{K\subseteq X\\ \text{compact}}} \wh{\coShv}(K;\cat C),
\]
which is functorial in all maps.
\end{defn}

\begin{rem}
We have
\[
\bigl(\wh{\coShv}(X;\cat C)\bigr)^{\omega} = \Fun^{sL}\bigl(\Sp, \wh{\coShv}(X;\cat C)\bigr) = \Fun^{sL}(\Shv(X);\cat C) \subseteq \coShv(X;\cat C),
\]
which is a full subcategory on all cosheaves $\sheaf F$ such that $\sheaf F(U)\to \sheaf F(V)$ is compact for $U\Subset V$.
\end{rem}

\begin{prop}
Assume that the topos $\Shv(X;\Anima)$ is of locally constant shape (e.g., if $X$ is hypercomplete and sublocally contractible, or is ANR). Equivalently, the functor $p^*\colon \Anima \to \Shv(X,\Anima)$ has a left adjoint $p_{\natural}$ (in addition to its obvious right adjoint $p_*$).

Then $\Shv(X; \Sp)$ is proper. If $X$ is countable at $\infty$, then it is $\omega_1$-compact.
\end{prop}
\begin{proof}
We need to show that the evaluation
\[
\Shv(X;\Sp) \otimes \Shv(X;\Sp) \xrightarrow{\otimes=\Delta^*} \Shv(X;\Sp) \xrightarrow{p_!} \Sp
\]
is strongly left adjoint. Since $\Delta^*$ is strongly left adjoint, we have to show that $p_!$ is strongly left adjoint. As $p_!$ is dual to $p^*$, this is the case if and only if $p^*$ admits a left adjoint.
\end{proof}

\begin{cor}
Under these assumptions we have
\begin{align*}
K^{\mathrm{cont}}(\wh{\coShv}(X;\cat C)) &= {KK}^{\mathrm{cont}}(\Shv(X); \cat C) \stackrel{\text{def}}{=} \map_{\Mot}(\cat U\Shv(X), \cat U\cat C) \\
&\cong \H_{\mathrm{lf}}(X;K^{\mathrm{cont}}(\cat C)) = p_*p^!K^{\mathrm{cont}}(\cat C) \\
&\cong \Pi_\infty X \otimes K^{\mathrm{cont}}(\cat C) \qquad\text{(if $X$ is compact)}
\end{align*}
where $\cat U$ is the universal localizing invariant.
\end{cor}

\begin{prop}
Let $X\in \LCHaus$ be $\sigma$-compact and of stably locally constant shape.
There is a canonical compact object
\[
\chi^{\mathrm{loc}} \in \wh{\coShv}(X;\Sp)
\]
given as $p_\natural = p_!(\blank\otimes \omega_X) \colon \Shv(X;\Sp) \to \Sp$. As a cosheaf it is given by $U\mapsto \Sigma^\infty_+\Pi_\infty U$.
\end{prop}

\begin{cor}
If $X$ is compact, then $\chi^{\mathrm{loc}} \in \Pi_\infty X\otimes K^{\mathrm{cont}}\Sp$.
\end{cor}

\begin{question}
Is there a \enquote{nice} description of $K_0^{\mathrm{cont}}(\cat C)$?
\end{question}

\begin{thm}[Bartels--N.]
There is a strongly continuous functor 
\[
A\colon \wh{\coShv}_{\mathrm{cs}}(X;\cat C) \to \cat C^{\Pi_\infty X} = \Loc(X;\cat C)
\]
with the following properties:
\begin{enumerate}[(i)]
\item It induces the assembly map on K-theory.
\item It takes $\chi^{\mathrm{loc}}$ to $\chi = \SS$ if $X$ is compact and $\cat C = \Sp$.
\end{enumerate}
\end{thm}
\begin{proof}[Proof/Construction]
We prove (i) under the assumption that $X$ is compact. Then
\begin{align*}
\wh{\coShv}(X;\cat C) &= \iHom^{\mathrm{dual}}(\Shv(X);\cat C) \xrightarrow{\otimes \Sp^{\Pi_\infty X}} \iHom\bigl(\Shv(X)\otimes \Sp^{\Pi_\infty X}, \cat C^{\Pi_\infty X}\bigr) \\
&\xrightarrow{D^*} \iHom^{\mathrm{dual}}(\Sp, \cat C^{\Pi_\infty X}) = \cat C^{\Pi_\infty X},
\end{align*}
where $D\colon \Sp \to \Shv(X)\otimes \Sp^{\Pi_\infty X}$ is left adjoint to
\[
\Shv(X)\otimes \Sp^{\Pi_\infty X} \xrightarrow{\psi^*} \Shv(X) \otimes \Shv(X)\xrightarrow{\Delta^*} \Shv(X) \to \Sp.
\]
\end{proof}










\end{document}
