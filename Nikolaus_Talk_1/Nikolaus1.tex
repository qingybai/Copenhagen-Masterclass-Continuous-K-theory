\documentclass[draft]{amsart}
\usepackage[utf8]{inputenc}
\usepackage[T1]{fontenc}
\usepackage{amssymb,amsthm}
\usepackage{mathtools}
\usepackage{tikz-cd}
\usepackage{stmaryrd} % \mapsfrom
\usepackage[shortlabels]{enumitem}
\usepackage{geometry}
\usepackage[final,kerning,spacing]{microtype}\frenchspacing
\usepackage{csquotes}
\usepackage[final]{hyperref}
\usepackage[capitalize]{cleveref}
\usepackage[inline]{showlabels}

\newcommand{\todo}[1]{\textcolor{red}{TODO: #1}}

\newcommand{\NN}{\mathbb{N}}
\newcommand{\ZZ}{\mathbb{Z}}
\newcommand{\QQ}{\mathbb{Q}}
\newcommand{\RR}{\mathbb{R}}
\newcommand{\CC}{\mathbb{C}}
\newcommand{\EE}{\mathbb{E}}
\newcommand{\FF}{\mathbb{F}}
\renewcommand{\AA}{\mathbb{A}}
\renewcommand{\SS}{\mathbb{S}}
\newcommand{\GG}{\mathbb{G}}
\newcommand{\PP}{\mathbb{P}}

\renewcommand{\H}{\mathrm{H}}
\newcommand{\D}{\mathrm{D}}
\newcommand{\R}{\mathrm{R}}
\renewcommand{\O}{\mathcal{O}}

\newcommand{\jhat}{\hat\jmath}

\newcommand{\set}[2]{\left\{#1\,\middle|\,#2\right\}}
\newcommand{\ol}[1]{\overline{#1}}
\newcommand{\ul}[1]{\underline{#1}}
\newcommand{\wh}[1]{\widehat{#1}}
\newcommand{\wt}[1]{\widetilde{#1}}
\newcommand{\cat}[1]{\mathcal{#1}}
\newcommand{\sheaf}[1]{\mathcal{#1}}

\renewcommand{\setminus}{\smallsetminus}
\renewcommand{\emptyset}{\vanothing}
\newcommand{\ssubset}{\Subset} 
\newcommand{\ssupset}{\Supset} 
\newcommand{\op}{\mathrm{op}}

\newcommand{\too}{\longrightarrow}
\newcommand{\xto}[1]{\mathbin{\xrightarrow{#1}}}
\newcommand{\isoto}{\mathbin{\xrightarrow{\sim}}}
\newcommand{\injto}{\mathbin{\hookrightarrow}}
\newcommand{\injtoo}{\mathbin{\longhookrightarrow}}
\newcommand{\epito}{\mathbin{\to\kern-.8em\to}}
\newcommand{\epitoo}{\mathbin{\too\kern-.8em\to}}

\renewcommand{\Pr}{\mathrm{Pr}}
\newcommand{\Stab}{\mathrm{Stab}}

\newcommand{\blank}{-} % todo

\DeclareMathOperator{\Spec}{Spec}
\DeclareMathOperator{\Hom}{Hom}
\DeclareMathOperator{\Anima}{Ani}
\DeclareMathOperator{\fib}{fib}
\DeclareMathOperator{\End}{End}
\DeclareMathOperator{\Spa}{Spa}
\DeclareMathOperator{\dlog}{dlog}
\DeclareMathOperator{\SingularSupport}{SS}
\DeclareMathOperator{\Ind}{Ind}
\DeclareMathOperator{\Map}{Map}
\DeclareMathOperator{\Open}{Open}
\DeclareMathOperator{\Seminorm}{Seminorm}
\DeclareMathOperator{\map}{map}
\DeclareMathOperator{\Shv}{Shv}
\DeclareMathOperator{\Nuc}{Nuc}
\DeclareMathOperator{\Cat}{Cat}
\DeclareMathOperator{\CompAss}{CompAss}
\DeclareMathOperator{\Sp}{Sp}
\DeclareMathOperator{\Fun}{Fun}
\DeclareMathOperator{\Mod}{Mod}
\DeclareMathOperator{\Perf}{Perf}
\DeclareMathOperator{\CompHaus}{CompHaus}
\DeclareMathOperator{\Spf}{Spf}
\DeclareMathOperator{\PSh}{PSh}
\DeclareMathOperator{\Corr}{Corr}
\DeclareMathOperator{\Pro}{Pro}
\DeclareMathOperator{\Image}{Im}
\DeclareMathOperator{\Tate}{Tate}
\DeclareMathOperator{\Ext}{Ext}
\DeclareMathOperator{\Vect}{Vect}
\DeclareMathOperator{\Spv}{Spv}
\DeclareMathOperator{\Calk}{Calk}
\DeclareMathOperator{\QCoh}{QCoh}
\DeclareMathOperator{\GL}{GL}
\DeclareMathOperator*{\indinjlim}{\text{``}\varinjlim\text{''}} % todo
\DeclareMathOperator*{\indprojlim}{\text{``}\varprojlim\text{''}} % todo
\DeclareMathOperator*{\colim}{colim}

\newtheorem{thm}{Theorem}[section]
\newtheorem{prop}[thm]{Proposition}
\newtheorem{lem}[thm]{Lemma}
\newtheorem{cor}[thm]{Corollary}

\theoremstyle{definition}
\newtheorem{defn}[thm]{Definition}
\newtheorem{rem}[thm]{Remark}
\newtheorem{properties}[thm]{Properties}
\newtheorem{fact}[thm]{Fact}
\newtheorem{ex}[thm]{Example}
\newtheorem{exercise}[thm]{Exercise}
\newtheorem{warning}[thm]{Warning}
\newtheorem*{claim}{Claim}
\newtheorem*{notation}{Notation}

\begin{document}
\title{Continuous K-Theory and Geometric Topology}
\author{Thomas Nikolaus}
\maketitle
\tableofcontents

\section{Talk 1}
Recall that $\Pr^L_{\mathrm{st}}$ is the category of presentable stable categories. We denote by $\otimes$ the Lurie tensor product; then $\Fun^L(\cat C, \cat D)$ is the internal Hom and the category $\Sp$ of spectra is the unit for $\otimes$.

\begin{defn}
$\cat C$ is \emph{dualizable} if $\cat C$ is dualizable in $(\Pr^L_{\mathrm{st}}, \otimes)$. The dual is then given by $\cat C^\vee = \Fun^L(\cat C, \Sp)$.
\end{defn}

\begin{defn}
\begin{enumerate}[(i)]
\item A functor $F\colon \cat D\to \cat E$ is called a \emph{homological epi} if the restriction $\Ind(\cat E)\to \Ind(\cat D)$ is fully faithful or, equivalently, if the induced functor $\Ind(\cat D)\to \Ind(\cat E)$ is a Bousfield localization.

\item A map $R\to S$ of ring spectra is a \emph{homological epi} if the following equivalent conditions hold:
\begin{itemize}
\item the induced functor $\Mod(R)^{\omega} \to \Mod(S)^\omega$ is a homological epi.
\item $\Mod(R) \to \Mod(S)$ is a Bousfield localization.
\item the restriction $\Mod(S) \to \Mod(R)$ is fully faithful.
\item $S\otimes_RS \isoto S$.
\item $S\sqcup_RS \isoto S$, where the pushout is taken in $\EE_1$-rings.
\item $I\otimes_RI \isoto I$, where $I = \fib(R\to S)$.
\end{itemize}

\item A non-unital ring spectrum $R$ is called \emph{H-unital} if $R^+ = R\oplus \SS \to \SS$ is a homological epi, where $\SS$ is the sphere spectrum.
\end{enumerate}
\end{defn}

\begin{defn}
\begin{enumerate}[(i)]
\item A map $x\to y$ in $\cat C$ is called \emph{(weakly) compact} if for every filtered colimit $d = \colim_i d_i$ and every map $y\to d$, the composite $x\to y\to d$ factors over some $d_{i_0}$:
\[
\begin{tikzcd}
x \ar[d,dashed] \ar[r] & y \ar[d] \\
d_{i_0} \ar[r] & \colim_i d_i.
\end{tikzcd}
\]

\item An object $x \in \cat C$ is called \emph{compactly exhaustible} if it can be written as
\[
x = \colim (x_0\to x_1\to \dotsb)
\]
with compact transition maps $x_i\to x_{i+1}$.

\item An object $x\in \cat C$ is called \emph{transfinitely compactly exhaustible} if it can be written as
\[
x = \colim_{i\in I} x_i,
\]
where $I$ is filtered, without terminal object, and \emph{antisymmetric} (i.e., the non-invertible morphisms in $I$ form an ideal; equivalently one-sided inverses are invertible), such that for every non-invertible map $i\to j$ in $I$, the induced map $x_i\to x_j$ is compact.
\end{enumerate}
\end{defn}

\begin{thm}\label{Nikolaus:dualizable}
For $\cat C \in \Pr^L_{\mathrm{st}}$ the following are equivalent:
\begin{enumerate}[(1)]
\item $\cat C$ is dualizable.

\item $\cat C$ is a retract in $\Pr^L_{\mathrm{st}}$ of a compactly generated stable category, i.e., $\cat C = \Ind(\cat C_0)$ for a small stable category $\cat C_0$.

\item $\cat C$ is the kernel of 
\[
\Ind(\cat D) \xrightarrow{\Ind(F)} \Ind(\cat E),
\]
where $\cat D, \cat E$ are small stable categories and $F\colon \cat D\to \cat E$ is an exact homological epi.

\item $\cat C$ is the kernel of 
\[
\Mod(R) \to \Mod(S)
\]
for a map $R\to S$ of ring spectra which is a homological epi. In this case we write $\cat C = \Mod(R,I)$.

\item $\cat C$ is the kernel of 
\[
\Mod(R^+) \to \Mod(\SS) = \Sp,
\]
where $R$ is an H-unital ring spectrum. In this case we write $\cat C = \Mod_\H(R)$.

\item The colimit functor $k\colon \Ind(\cat C) \to \cat C$ admits a left adjoint $\jhat$ (which is automatically fully faithful since the right adjoint $j$ of $k$ is fully faithful).

\item $\cat C$ is $\omega_1$-compactly generated and the colimit
\[
k\colon \Ind(\cat C^\omega_1) \to \cat C
\]
admits a left adjoint.

\item $\cat C$ is generated under colimits by compactly exhaustible objects.

\item Every object in $\cat C$ is transfinitely compactly exhaustible.

\item (AB6) Products in $\cat C$ distribute over filtered colimits: the natural map
\[
\colim_{(i_k)_k \in \prod_{k\in K}I_k} \prod_{k\in K} x_{k,i_k} \isoto
\prod_{k\in K} \colim_{i\in I_k} x_{k,i}
\]
is an isomorphism in $\cat C$.\footnote{Since we are in a stable category, we may equivalently replace products with limits. Moreover, one can assume that all $I_k$'s are the same.}
\end{enumerate}
\end{thm}
\begin{proof}[\enquote{Proofs}:]
If $\cat C$ is dualizable, we have a Bousfield localization $\Ind(\cat C^\kappa)\to \cat C$. We then obtain a lifting diagram
\[
\begin{tikzcd}
\Ind(\cat C^\kappa) \otimes \cat C^\vee \ar[r] & \cat C\otimes \cat C^\vee \\
& \Sp \ar[u,"\text{coev}"'] \ar[ul,dashed],
\end{tikzcd}
\]
where the lift exists, because a functor from $\Sp$ is uniquely determined by specifying the image of $\SS$. Reinterpreting, we obtain a lift
\[
\begin{tikzcd}
& \Ind(\cat C^\kappa) \ar[d] \\
\cat C \ar[ur,dashed] \ar[r,equals] & \cat C.
\end{tikzcd}
\]

For the implication (2)$\implies$(6), write $\cat C = \Ind(\cat C_0)$. Then we have
\[
\jhat = \Ind(\cat C_0\injto \cat C) \colon \cat C = \Ind(\cat C_0)\to \Ind(\cat C).
\]
To pass to retracts, use some abstract argument.

For the implication (6)$\implies$(2), use $\jhat\colon \cat C \to \Ind(\cat C)$.

Ad (7)$\implies$(3): Consider 
\[
\cat C \xrightarrow{\jhat} \Ind(\cat C^{\omega_1}) \to \Ind(\cat C^{\omega_1})/\cat C = \Ind(\Calk^{\mathrm{cont}}(\cat C)).
\]
To see the equality, use that $\cat C^{\omega_1} \xrightarrow{\mathrm{pr}} \Calk^{\mathrm{cont}}(\cat C)$ is a homological epi.

Finally, for (6)$\iff$(10), note that distributivity is equivalent to $k\colon \Ind(\cat C)\to \cat C$ preserving products. (This relies on the fact that the category of anima satisfies (AB6).)
\end{proof}

\begin{ex}
The category $\Shv(X)$ of sheaves with values in $\Sp$ on a locally compact space $X$ is generated by sheaves of the form $\Sigma^\infty_+ \ul{U}$, where $\ul U$ is the sheaf on $X$ represented by $U$ and $\Sigma^\infty_+\colon \Anima \to \Sp$ is the suspension functor. Every inclusion $U \injto V$ in $\Open(X)$ factors through a compact subset $K$, that is, $U \to K\to V$. It follows that $\Sigma^\infty_+ \ul{U} \to \Sigma^\infty_+\ul{V}$ is a compact map. Finally, $\Sigma^\infty_+\ul{U}$ is compactly exhaustible if $U$ is, and every $U$ is a filtered colimit of such. Therefore, $\Shv(X)$ is dualizable.
\end{ex}

\begin{properties}[of dualizable categories]
Let $\cat C, \cat D$ be dualizable categories.
\begin{enumerate}[(a)]
\item An object $x\in \cat C$ is compactly exhaustible if and only if $x$ is $\omega_1$-compact.

\item A functor $F\colon \cat C\to \cat D$ in $\Pr^L$ is strongly continuous (meaning that $F$ admits a right adjoint that preserves colimits) if and only if $F$ preserves compact morphisms, if and only if the following diagram commutes:
\[
\begin{tikzcd}
\Ind(\cat C) \ar[r,"\Ind(F)"] & \Ind(\cat D) \\
\cat C \ar[u,"\jhat"] \ar[r,"F"'] & \cat D \ar[ul,Rightarrow,shorten=4mm, "\sim"'] \ar[u,"\jhat"'].
\end{tikzcd}
\]

\item A morphism $x\to y$ in $\cat C$ is compact if it lifts to a map $jx \to \jhat y$ in $\Ind(\cat C)$. In this case, we define the space of \emph{compactly assembled maps} as the spectrum
\[
\map_{\cat C}^{\mathrm{ca}}(x,y)\coloneqq \map_{\Ind(\cat C)}(jx, \jhat y).
\]

\item If $x = \colim_{i\in I} x_i$ is $I$-compactly exhaustible, then $\jhat(x) = \colim_{i\in I} jx_i$ in $\Ind(\cat C)$.

\item Recall the resolution 
\[
\cat C \to \Ind(\cat C^{\omega_1}) \to \Ind(\Calk^{\mathrm{cont}}(\cat C)),
\]
and denote $p\colon \cat C^{\omega_1} \to \Calk^{\mathrm{cont}}(\cat C)$ the projection. Then
\[
\map_{\Calk(\cat C)}(px, py) = \map_{\cat C}(x,y) / \map_{\cat C}^{\mathrm{ca}} (x,y).
\]
\end{enumerate}
\end{properties}

\begin{defn}
We denote $\Pr^L_{\mathrm{dual}}$ the category of dualizable categories with strongly left adjoint functors.
\end{defn}

\begin{cor}
A left adjoint functor $\Shv(X) \to \cat D$ is strongly left adjoint if the corresponding cosheaf $\sheaf F\colon \Open(X) \to \cat D$ satisfies the following condition: for every $U\ssubset V$, the induced map $\sheaf F(U) \to \sheaf F(V)$ is a compact morphism in $\cat D$.
\end{cor}

\begin{defn}
We define the continuous K-theory of a dualizable category $\cat C$ as the fiber
\[
K^{\mathrm{cont}}(\cat C) = \fib\bigl(K(\cat C^{\omega_1}) \to K(\Calk^{\mathrm{cont}}(\cat C))\bigr) \cong \Omega K(\Calk^{\mathrm{cont}}(\cat C)).
\]
\end{defn}



\end{document}











