\documentclass[draft]{amsart}
\usepackage[utf8]{inputenc}
\usepackage[T1]{fontenc}
\usepackage{amssymb,amsthm}
\usepackage{mathtools}
\usepackage{tikz-cd}
\usepackage{stmaryrd} % \mapsfrom
\usepackage[inline,shortlabels]{enumitem}
\usepackage{geometry}
\usepackage[final,kerning,spacing]{microtype}\frenchspacing
\usepackage{csquotes}
\usepackage[final]{hyperref}
\usepackage[capitalize]{cleveref}
\usepackage[inline]{showlabels}

\newcommand{\todo}[1]{\textcolor{red}{TODO: #1}}

\newcommand{\NN}{\mathbb{N}}
\newcommand{\ZZ}{\mathbb{Z}}
\newcommand{\QQ}{\mathbb{Q}}
\newcommand{\DD}{\mathbb{D}}
\newcommand{\RR}{\mathbb{R}}
\newcommand{\CC}{\mathbb{C}}
\newcommand{\FF}{\mathbb{F}}
\renewcommand{\AA}{\mathbb{A}}
\renewcommand{\SS}{\mathbb{S}}
\newcommand{\GG}{\mathbb{G}}
\newcommand{\PP}{\mathbb{P}}

\renewcommand{\H}{\mathrm{H}}
\newcommand{\D}{\mathrm{D}}
\newcommand{\R}{\mathrm{R}}
\renewcommand{\L}{\mathrm{L}}
\renewcommand{\O}{\mathcal{O}}

\newcommand{\set}[2]{\left\{#1\,\middle|\,#2\right\}}
\newcommand{\ol}[1]{\overline{#1}}
\newcommand{\ul}[1]{\underline{#1}}
\newcommand{\wh}[1]{\widehat{#1}}
\newcommand{\wt}[1]{\widetilde{#1}}
\newcommand{\cat}[1]{\mathcal{#1}}
\newcommand{\sheaf}[1]{\mathcal{#1}}

\renewcommand{\setminus}{\smallsetminus}
\renewcommand{\emptyset}{\varnothing}
\newcommand{\ssubset}{\Subset} 
\newcommand{\ssupset}{\Supset} 
\newcommand{\op}{\mathrm{op}}

\newcommand{\too}{\longrightarrow}
\newcommand{\xto}[1]{\mathbin{\xrightarrow{#1}}}
\newcommand{\isoto}{\mathbin{\xrightarrow{\sim}}}
\newcommand{\injto}{\mathbin{\hookrightarrow}}
\newcommand{\injtoo}{\mathbin{\longhookrightarrow}}
\newcommand{\epito}{\mathbin{\to\kern-.8em\to}}
\newcommand{\epitoo}{\mathbin{\too\kern-.8em\to}}

\renewcommand{\Pr}{\mathrm{Pr}}
\newcommand{\Stab}{\mathrm{Stab}}

\newcommand{\blank}{-} % todo
\newcommand{\one}{\mathbf{1}}

\DeclareMathOperator{\Spec}{Spec}
\DeclareMathOperator{\Hom}{Hom}
\DeclareMathOperator{\coker}{coker}
\DeclareMathOperator{\fib}{fib}
\DeclareMathOperator{\Man}{Man}
\DeclareMathOperator{\Closed}{Closed}
\DeclareMathOperator{\id}{id}
\DeclareMathOperator{\Idem}{Idem}
\DeclareMathOperator{\CAlg}{CAlg}
\DeclareMathOperator{\Sym}{Sym}
\DeclareMathOperator{\Null}{Null}
\DeclareMathOperator{\CondAb}{CondAb}
\DeclareMathOperator{\End}{End}
\DeclareMathOperator{\Spa}{Spa}
\DeclareMathOperator{\iHom}{\ul{Hom}}
\DeclareMathOperator{\dlog}{dlog}
\DeclareMathOperator{\SingularSupport}{SS}
\DeclareMathOperator{\Ind}{Ind}
\DeclareMathOperator{\Map}{Map}
\DeclareMathOperator{\ev}{ev}
\DeclareMathOperator{\Open}{Open}
\DeclareMathOperator{\Seminorm}{Seminorm}
\DeclareMathOperator{\Shv}{Shv}
\DeclareMathOperator{\Nuc}{Nuc}
\DeclareMathOperator{\Cat}{Cat}
\DeclareMathOperator{\CompAss}{CompAss}
\DeclareMathOperator{\Sp}{Sp}
\DeclareMathOperator{\Fun}{Fun}
\DeclareMathOperator{\Mod}{Mod}
\DeclareMathOperator{\Perf}{Perf}
\DeclareMathOperator{\CompHaus}{CompHaus}
\DeclareMathOperator{\Spf}{Spf}
\DeclareMathOperator{\PSh}{PSh}
\DeclareMathOperator{\Corr}{Corr}
\DeclareMathOperator{\Pro}{Pro}
\DeclareMathOperator{\Image}{Im}
\DeclareMathOperator{\Tate}{Tate}
\DeclareMathOperator{\Fin}{Fin}
\DeclareMathOperator{\Ext}{Ext}
\DeclareMathOperator{\Vect}{Vect}
\DeclareMathOperator{\Spv}{Spv}
\DeclareMathOperator{\Calk}{Calk}
\DeclareMathOperator{\QCoh}{QCoh}
\DeclareMathOperator{\GL}{GL}
\DeclareMathOperator*{\indinjlim}{\text{``}\varinjlim\text{''}} % todo
\DeclareMathOperator*{\indprojlim}{\text{``}\varprojlim\text{''}} % todo
\DeclareMathOperator*{\colim}{colim}

\newtheorem{thm}{Theorem}[section]
\newtheorem{prop}[thm]{Proposition}
\newtheorem{mainconj}[thm]{Main Conjecture}
\newtheorem{lem}[thm]{Lemma}
\newtheorem{cor}[thm]{Corollary}

\theoremstyle{definition}
\newtheorem{defn}[thm]{Definition}
\newtheorem{rem}[thm]{Remark}
\newtheorem{question}[thm]{Question}
\newtheorem{fact}[thm]{Fact}
\newtheorem{ex}[thm]{Example}
\newtheorem{exercise}[thm]{Exercise}
\newtheorem{warning}[thm]{Warning}
\newtheorem*{claim}{Claim}
\newtheorem*{notation}{Notation}

\begin{document}
\title{Three persectives on Deligne cohomology}
\author{Dustin Clausen}
\maketitle
\tableofcontents


\section{Talk 1: Deligne cohomology}

Deligne cohomology is a cohomology theory for complex manifolds which refines the usual singular/sheaf cohomology $\H^*(M;\ZZ)$ by including some differential form data.

\subsection{Motivation}
Let $V$ be a holomorphic vector bundle over a complex manifold $M$. Then we get a complex topological vector bundle on the topological space $M$, hence a Chern class $c_p(V) \in \H^{2p}(M;\ZZ)$.

We want refined coefficients $\ZZ(p)_{\D}$ which maps to $\ZZ$ such that $c_p(V) \in \H^{2p}(M;\ZZ)$ functorially (with respect to pullback of vector bundles) lifts to $\H^{2p}(M; \ZZ(p)_{\D})$.

\begin{rem}
Why do the coefficients depend on $p$? Note that $c_1(\sheaf L) \in \H^{2}(M;\ZZ)$, where $\ZZ$ should be identified with $\H_1(\CC^\times; \ZZ) \cong 2\pi i\ZZ \subseteq \CC$.

Similarly, we should have $c_p(V) \in \H^{2p}(M; (2\pi i)^{p}\ZZ)$.
\end{rem}

To define $\ZZ(p)_{\D}$, let us look at first Chern classes:
\[
c_1(V) = c_1(\det(V)).
\]
One description is the following: The short exact sequence
\[
0\to 2\pi i\ZZ \to \O \xrightarrow{\exp} \O^\times \to 1.
\]
gives a map $\H^1(M;\O^\times) \to \H^2(M;2\pi i\ZZ)$. This suggests setting
\[
\ZZ(1)_{\D} = \O^\times[-1].
\]

\textbf{Reinterpretation:} We have a homotopy pullback (because the cofibers on both horizontal maps identify with $\O$):
\[
\begin{tikzcd}
\O^\times{[-1]} \ar[r,"\partial"] \ar[d,"\dlog(f) = \frac{\mathrm{d} f}{f}"'] & 2\pi i\ZZ \ar[d] \\
{[0\to \Omega^1\to \Omega^2\to\dotsb]} \ar[r] & {[\Omega^0 \to \Omega^1 \to \dotsb]}
\end{tikzcd}
\]
where $\Omega_{\mathrm{dR}} \coloneqq (\Omega^\bullet,d)$ is the holomorphic de\,Rham complex.

\begin{defn}
For $p\in \ZZ$, $p\ge0$, define
\[
\begin{tikzcd}
\ZZ(p)_{\D} \ar[d] \ar[dr,phantom, very near start, "\lrcorner"] \ar[r] & (2\pi i)^p\ZZ \ar[d] \\
F^p\Omega_{\mathrm{dR}} = {[0\to \dotsb\to 0\to \Omega^p\to \Omega^{p+1}\to \dotsb]} \ar[r] & {[\Omega^0\to \Omega^1\to\dotsb]} = \Omega_{\mathrm{dR}},
\end{tikzcd}
\]
where on the left $\Omega^p$ sits in homological degree $-p$.
\end{defn}

\begin{ex}
\begin{enumerate}[(i)]
\item $\ZZ(0)_{\D} = \ZZ$;
\item $\ZZ(1)_{\D} = \O^\times[-1]$, which implies that there is a tautological Chern class $c_1(\sheaf L) \in \H^2(M; \ZZ(1)_{\D})$.

\item $\bigoplus_p \ZZ(p)_{\D}$ is a graded commutative ring, i.e., there is a map $\ZZ(p)_\D \otimes \ZZ(q)_{\D} \to \ZZ(p+q)_{\D}$.

\item Pullback functoriality: for each $M\to N$ there is a map $\H^*(N; \ZZ(p)_{\D}) \to \H^*(M;\ZZ(p)_{\D})$.

\item Projective bundle formula: Let $V \to M$ be a vector bundle of dimension $d$ and consider its projectivization $\PP(V) \to M$. Then
\[
\bigoplus_p\R\Gamma(\PP(V); \ZZ(p)_{\D})
\]
is graded free of rank $d$ over $\bigoplus_p \R\Gamma(M;\ZZ(p)_{\D})$ on $1, c_1(\O(1)), \dotsc, c_1(\O(1))^{d-1}$.

By Grothendieck, we can expand $c_1(\O(1))^d$ in terms of previous powers, and the coefficients define the higher Chern classes $c_p(V) \in \H^{2p}(M; \ZZ(p)_{\D})$.
\end{enumerate}
\end{ex}

\begin{ex}
If $p\le 0$, then $\ZZ(p)_{\D} = (2\pi i)^p\ZZ$, which is \enquote{purely topological}.

If $p>\dim M$, then $\ZZ(p)_{\D} = \CC/(2\pi i)^p \ZZ[-1]$.
\end{ex}

\begin{rem}
Let $M$ be Stein (i.e., a closed submanifold $M\injto \CC^N$). Then $\Omega^i$ is acyclic and hence $\H^p(F^p\Omega_{\mathrm{dR}}) = \Omega^p_{\mathrm{cl}}$ is the space of holomorphic closed $p$-forms (these are huge vector spaces!).
\end{rem}

If $M$ is compact, then $\dim_{\CC} (\bigoplus_{i,p} \H^i(M; \Omega^p)) < \infty$ and hence the Deligne cohomlogy groups are always built out of $\ZZ$'s and $\CC$'s by extensions and quotients.

If $M$ is compact K\"ahler (e.g., a smooth projective variety over $\CC$), then the map
\[
\H^*(M; F^p\Omega_{\mathrm{dR}}) \to \H^*(M; \Omega_{\mathrm{dR}}) \simeq \H^*(M;\CC)
\]
is \emph{injective}. The image is $F^p\H^*(M;\CC)$ in the Hodge filtration. Thus, we have a short exact sequence
\[
0\to \frac{\H^{i-1}(M;\CC)}{F^p\H^{i-1}(M;\CC) + \H^{i-1}(M; (2\pi i)^{p-1}\QQ)} \to 
\H^i(M; \QQ(p)_{\D}) \to F^p\H^i(M;\CC) \cap \H^i(M; (2\pi i)^p\QQ) \to 0,
\]
which we view as an extension of something \enquote{discrete} by something \enquote{continuous}.

When $i = 2p$, the left hand side is $J^p(M)_{\QQ}$, that is, Griffith's intermediate Jacobian; for $p=1$ this is the usual Jacobian.

\subsection{Applications} 
\begin{enumerate}[(i)]
\item One application, the intermediate Jacobians $J^p(M)$, were already mentioned.

\item Secondary characteristic classes of flat bundles
\[
\begin{tikzcd}
c_p(V) \ar[r,phantom,"\in"] &[-2em]  \H^{2p-1}(B\GL_n(\CC)^{\delta}, \CC/(2\pi i)^p\ZZ), \ar[d,"\partial"] & p>0 \\
& \H^{2p}(B\GL_n(\CC)^{\delta}, (2\pi i)^p\ZZ),
\end{tikzcd}
\]
also called the Chern--Simons invariants.

\item \textbf{Arithmetic:} Let $\cat{X} \to \Spec(\ZZ)$ be a regular proper scheme over $\ZZ$. Then $\cat X$ should not be thought of as compact, because $\Spec(\ZZ)$ is not compact ($\Spec(\ZZ)$ corresponds to the affine line $\Spec(\FF_p[T]) = \AA^1_{\FF_p}$. 

There is a small neighborhood around $\infty \in \PP^1_{\FF_p}$ corresponding to $\FF_p[T] \to \FF_p((T^{-1}))$ (which which should be thought of as corresponding to the inclusion $\{0\} \to \RR$).

On $\cat X$, we should consider not just cohomology, but \enquote{compactly supported cohomology}, namely
\[
\fib\bigl(\R\Gamma(\cat X) \to \R\Gamma(\cat X(\CC)/C_2)\bigr).
\]
The idea of Arakelov theory is the following: If \enquote{cohomology} means motivic cohomology, then $\R\Gamma(\cat X(\CC))$ should be Deligne cohomology.
\end{enumerate}

\begin{ex}
Consider the motivic cohomology
\[
\H^i_{\mathrm{M}}(\Spec(\ZZ);\QQ(p)) \to \H^i(*_{\CC}; \QQ(p)_{\D})^{C_2}.
\]
In weights $p>1$, (by Borel) the left hand side is non-zero if and only if $i=1$ and $p$ is odd, in which case it is a one-dimensional $\QQ$-vector space. The right hand side is non-zero if and only if $i=1$ and $p$ is odd, in which case it is isomorphic to $\CC/(2\pi i)^p\QQ \xrightarrow{\mathrm{Re}} \RR$. The image of the induced map $\QQ \to \RR$ can be identified with $\pi^? \zeta(p)\QQ$, where $\zeta$ is the Riemann $\zeta$-function.
\end{ex}

\subsection{Goals of this lecture series} 
\begin{enumerate}[(a)]
\item We want to make precise the idea that Deligne cohomology is the analog of motivic cohomology for complex manifolds.

\item A second goal is to understand the Hodge conjecture: For a complex smooth projective manifold $M$ we have:
\[
\begin{tikzcd}
K_0(\Vect(M))_{\QQ} \ar[dr,twoheadrightarrow] \ar[r,"\mathrm{ch}"] & \bigoplus_p \H^{2p}(M; \QQ(p)) \ar[d,twoheadrightarrow] \\
& \bigoplus_p \mathrm{Hdg}^p(M)_{\QQ}.
\end{tikzcd}
\]

We want to modify $K_0(\Vect(M))$ using continuous K-theory to get a theory where it is reasonable to conjecture that $K_0(\Nuc(M))_\QQ \isoto \bigoplus_p \H^{2p}(M;\QQ(p))$.

\item A third goal is to make Riemann--Roch more transparent.
\end{enumerate}

\section{Talk 2: Complex manifolds from a new perspective (joint with P. Scholze)}

Let $M$ be a complex manifold. The Deligne cohomology of $M$ was defined as a complex $\ZZ(p)_{\D}$ of sheaves, for any $p\in\ZZ$, given by the pullback
\[
\begin{tikzcd}
\ZZ(p)_{\D} \ar[d] \ar[dr,phantom, very near start, "\lrcorner"] \ar[r] & (2\pi i)^p\ZZ \ar[d] \\
F^p\Omega_{\mathrm{dR}} \ar[r] & \Omega_{\mathrm{dR}}.
\end{tikzcd}
\]

\textbf{Main goal:} \enquote{Fix} the fact that
\[
K_0(\Vect(M)) \to \bigoplus_p \H^{2p}(M; \ZZ(p)_\D)
\]
is far from being an isomorphism by replacing $K(\Vect(M))$ with $K^{\mathrm{cont}}(\Nuc(M))$. Here, $\Nuc(M)$ is some version of $D_{\mathrm{qc}}(M)$.

\textbf{Idea:} make it more like scheme theory: Start with a class of rings, $R$, which will determine everything: there is 
\begin{itemize}
\item an underlying topological space,
\item a structure sheaf of holomorphic functions,
\item a de\,Rham complex,
\item \ldots.
\end{itemize}

The basic example of an $R$ will be
\[
\O^{\mathrm{hol}}(\DD) = \set{\sum_n c_n T^n}{\text{$\exists r>1$ such that $\sum_n \lvert c_n\rvert r^n \to \infty$}},
\]
where $\DD \subseteq \CC$ is the \emph{closed} unit disk and $\O^{\mathrm{hol}}$ denotes the functions which are holomorphic in a neighborhood of $\DD$.

\textbf{Good news!} The abstract algebra $\O^{\mathrm{hol}}(\DD)$ determines $\DD$. More precisely, there are mutually inverse maps
\[
\begin{tikzcd}[column sep=5em]
\DD \ar[r,bend right,"x\mapsto \ev_x"'] & \Hom_{\CC}(\O^{\mathrm{hol}}(\DD), \CC) \ar[l,bend right, "\varphi(T) \mapsfrom \varphi"'] 
\end{tikzcd}
\]
\begin{proof}
We need: 
\begin{enumerate}[(i)]
\item $\varphi(T) \in \DD$. Suppose $\lambda\in \CC\setminus \DD$. Then $\frac1{T-\lambda} \in \O^{\mathrm{hol}}(\DD)$, that is, $T-\lambda$ is a unit, hence so is $\varphi(T-\lambda) = \varphi(T) - \varphi(\lambda) \neq 0$, which is a contradiction.

\item $\varphi$ is determined by $\varphi(T)$. This follows from the claim that
\[
\lim_{N\to \infty} \varphi\Bigl( \sum_{n\le N} c_n T^n\Bigr) = \varphi\Bigl(\sum_n c_n T^n\Bigr).
\]
The same argument as above shows that $\varphi(\sum_{n>N} c_nT^n) \in \varepsilon \DD$ for $N$ such that $\sum_{n>N} \lvert c_n\rvert <\varepsilon$.
\end{enumerate}
\end{proof}

\begin{defn}
Let $R$ be a $\CC$-algebra. Define
\[
\cat M_B(R) = \Hom_{\CC}(R, \CC) \subseteq \prod_{f\in R} \CC
\]
with the product topology.\footnote{The subscript \enquote{B} stands for Betti or Berkovich.}
\end{defn}

\begin{claim}
Both maps above are continuous.
\end{claim}
The proof of the claim is (more or less) obvious. We now have that
\[
\DD = \cat M_{B}(\O^{\mathrm{hol}}(\DD)).
\]

\textbf{Bad news:} We cannot get a structure sheaf, de\,Rham cohomology etc. just from the abstract $\CC$-algebra structure on $\O^{\mathrm{hol}}(\DD)$.
The most basic reason is that
\[
\O^{\mathrm{hol}}(\DD) \otimes_{\CC} \O^{\mathrm{hol}}(\DD) \neq \O^{\mathrm{hol}}(\DD^2),
\]
where $\otimes_{\CC}$ is the abstract tensor product.

\textbf{Solution:} remember the topological vector space structure on $\O^{\mathrm{hol}}(\DD)$ and use the \emph{completed} tensor product $\otimes_{\CC}$. Or rather, use a category-friendly version thereof.

Concretely, this means that \enquote{topological} vector spaces are replaced with \enquote{light condensed} $\CC$-vector spaces; then completeness corresponds to \enquote{gaseous}.

\begin{defn}
A \emph{light condensed abelian group} is a presheaf of abelian groups on $\Pro(\Fin)^{\mathrm{light}}$, the category of countable inverse limits of finite sets, satisfying descent with respect to 
\begin{enumerate*}[(1)]
\item finite coproducts and
\item surjections $S\epito T$.
\end{enumerate*}

\textit{Exercise:} We have that $\NN\cup\{\infty\}$ lies in $\Pro(\Fin)^{\mathrm{light}}$.
\end{defn}

\begin{ex}
For $\CC$, there is a light condensed ring given by $S\mapsto \cat C^0(S,\CC)$, where $\cat C^0(S,\CC)$ denotes set of continuous functions from $S$ to $\CC$.
\end{ex}

We can now consider the symmetric monoidal category
\[
\bigl(\Mod_{\CC}(\CondAb^{\mathrm{light}}), \otimes_{\CC}).
\]
We need to pass to a full subcategory of \enquote{complete} objects.

\textbf{Idea:} Completeness of $M$ corresponds to the following property: if $m_0, m_1, m_2, \dotsc$ is a null sequence in $M$, then we can form $\sum_n m_n \cdot (1/2)^n \in M$.

Here, a \emph{null sequence} is a map from $\ZZ[\NN\cup \{\infty\}]/\ZZ\infty$, the free condensed abelian group on a null sequence.

\begin{defn}
A module $M\in \Mod_{\CC}(\CondAb^{\mathrm{light}})$ is called \emph{gaseous} if the map
\[
\Null(M)\coloneqq \iHom(\ZZ[\NN\cup\{\infty\}]/\ZZ\infty, M) \xrightarrow[\simeq]{1-T\cdot \frac12} \iHom(\ZZ[\NN\cup\{\infty\}]/\ZZ\infty, M),
\]
where $T$ is induced by the shift map $\NN\to \NN$, $n\mapsto n+1$.
\end{defn}

\begin{rem}
The condition for $\frac12$ is equivalent to the condition for any $\lambda$ with $0< \lvert\lambda\rvert < 1$.
\end{rem}

\begin{thm}
The full subcategory $\Mod_{\CC^{\mathrm{gas}}} \subseteq \Mod_{\CC}(\CondAb^{\mathrm{light}})$ of gaseous $\CC$-vector spaces is abelian, closed under all colimits, limits, extensions, all $\R^i\varprojlim$, $\L^i\varinjlim$ and $\R^i\iHom(X,\blank)$, for all $X\in \Mod_{\CC}(\CondAb^{\mathrm{light}})$.
\end{thm}
\begin{proof}
Everything follows from the (interesting) fact that $\ZZ[\NN\cup\{\infty\}]/\ZZ\infty$ is \emph{(internally) projective} in $\Mod_{\CC}(\CondAb^{\mathrm{light}})$.
\end{proof}

\textbf{Upshot:} 
\begin{enumerate}[(i)]
\item There exists a left adjoint 
\[
(\blank)^{\mathrm{gas}}\colon \Mod_{\CC}(\CondAb^{\mathrm{light}}) \to \Mod_{\CC^{\mathrm{gas}}}
\]
to the inclusion,
\item There exists a symmetric monoidal structure on $\Mod_{\CC^{\mathrm{gas}}}$ making $(\blank)^{\mathrm{gas}}$ symmetric monoidal.

\item There is a derived analog of everything.
\end{enumerate}

\begin{ex}
Any Banach space over $\CC$ is gaseous. In particular, $\O^{\mathrm{hol}}(\DD)$ is gaseous (it is a filtered union of $\ell^1$-spaces).
\end{ex}

\begin{thm}
The ring $\O^{\mathrm{hol}}(\DD)$ is flat with respect to $\blank\otimes_{\CC^{\mathrm{gas}}}\blank$ and
\[
\O^{\mathrm{hol}}(\DD) \otimes_{\CC^{\mathrm{gas}}} M = \varinjlim \Null(M).
\]
\end{thm}
\begin{proof}
Use trace class map tricks.
\end{proof}

The rings $R$ that we consider are objects of $\CAlg(D_{\ge0}(\CC^{\mathrm{gas}}))$, which we call \emph{gaseous $\CC$-algebras}.

\begin{defn}
Let $R$ be a gaseous $\CC$-algebra.
\begin{enumerate}[(a)]
\item An element $f\in \pi_0R(*)$ is called \emph{topologically nilpotent} if there is a factorization
\[
\begin{tikzcd}
\CC[T] \ar[d] \ar[r,"T\mapsto f"] & R \\
\CC[\NN\cup\{\infty\}/\infty] \ar[ur,dashed,"\exists"']
\end{tikzcd}
\]
of condensed rings.

\item $R$ is called \emph{pointwise bounded} if for all $f\in R$ (meaning: $f\in \pi_0R(*)$), there exists $\lambda\in \CC$ with $0<\lvert \lambda\rvert <1$ such that $\lambda f$ is topologically nilpotent.
\end{enumerate}
\end{defn}

\begin{defn}[Rodriguez-Camargo]
Let $S\in \Pro(\Fin)^{\mathrm{light}}$. Then $f\in R(S)$ is \emph{uniformly topologically nilpotent} if there exists a factorization
\[
\begin{tikzcd}
\CC\Bigl[\bigsqcup_d \Sym^dS\Bigr] \ar[d] \ar[r,"f"] & R \\
\CC\Bigl[\Bigl(\bigsqcup_d \Sym^dS\Bigr)\cup\{\infty\}/\infty\Bigr] \ar[ur,dashed, "\exists"'].
\end{tikzcd}
\]

$R$ is called \emph{bounded} if for all $S$ and all $f\in R(S)$, there exists $\lambda\in \CC$ with $0 <\lvert \lambda\rvert < 1$ such that $\lambda f$ is uniformly topologically nilpotent.
\end{defn}

\begin{thm}[Rodriguez-Camargo]
$\CAlg(D_{\ge0}(\CC^{\mathrm{gas}}))^{\mathrm{bded}} \subseteq \CAlg(D_{\ge0}(\CC^{\mathrm{gas}}))$ is closed under all colimits and finite limits.
\end{thm}

\begin{ex}
Any Banach algebra $R$ over $\CC$ is bounded. In particular, $\O^{\mathrm{hol}}(\DD)$ is bounded.
\end{ex}

\begin{thm}
If $R$ is pointwise bounded, then $\cat M_B(R(*))$ is compact Hausdorff and
\[
D(R) \coloneqq \Mod_R(D(\CC^{\mathrm{gas}}))
\]
localizes along $\cat M_B(R(*))$ (the abstract $\CC$-algebra underlying $\cat M_B(R)$).
\end{thm}

\section{Talk 3: Quasicoherent sheaves in complex geometry (joint with P. Scholze)}
Recall, we considered $R \in \CAlg(D_{\ge0}(\CC^{\mathrm{gas}}))^{\mathrm{bded}}$, meaning that for all $f\in R$ there exists $\lambda\in \CC^\times$ such that $\lambda f$ is topologically nilpotent and similarly for $f\in R(S)$ with $S \in \Pro(\Fin)^{\mathrm{light}}$.\footnote{An analogous definition was considered by Ralf  Meyer.}

\begin{thm}\label{Clausen:Gelfand}
The category $D(R) \coloneqq \Mod_R(D(\CC^{\mathrm{gas}}))$ localizes on the compact Hausdorff space $\cat M_B(R(*)) = \Hom_{\CC}(R(*), \CC)$, called the \emph{Gelfand spectrum}.
\end{thm}

Note that under the boundedness condition we even have a closed embedding
\[
\cat M_B(R(*)) \subseteq \prod_{f\in R(*)} \CC_{\lvert\cdot\rvert \le C_f},
\]
where $C_f \in \RR_{>0}$ depends on $f$.

\begin{ex}
If $R = \O^{\mathrm{hol}}(\DD^n)$, then $\cat M_B(R(*)) = \DD^n$.
\end{ex}

\begin{defn}[Balmer--Krause--Stevenson]
Let $\cat C \in \CAlg(\Pr^L)$. Define a full subcategory
\[
\Idem(\cat C) \subseteq \CAlg(\cat C)
\]
consisting of the \emph{idempotent} algebras, i.e., algebras $R$ such that $\one\to R$ induces an isomorphism $R = \one\otimes R \isoto R\otimes R$ (equivalently, $m\colon R\otimes R\isoto R$ is an isomorphism).
\end{defn}

\begin{prop}
The category $\Idem(\cat C)$ is a poset.\footnote{This means that the anima of maps is either empty or contractible.}

Moreover, $\Idem(\cat C)$ has arbitrary colimits and finite limits, which are calculated as follows:
\begin{enumerate}[(1)]
\item sifted colimits are calculated in $\cat C$.
\item finite coproducts are calculated by $\otimes$.
\item pullbacks are computed as the fiber product 
\[
\begin{tikzcd}
\text{(pullback)} \ar[d] \ar[r] \ar[dr,phantom, very near start, "\lrcorner"] & R \ar[d] \\
S \ar[r] & S\otimes R.
\end{tikzcd}
\]
\end{enumerate}
Moreover, $\Idem(\cat C)$ is a \emph{locale} (i.e., it satisfies the same properties as open subsets of a topological space).
\end{prop}

A more precise version of \cref{Clausen:Gelfand} is the following:
\begin{thm}
There exists a map of posets (locales$^{\op}$?)
\[
\Closed(\cat M_B(R)) \to \Idem(D(R))^\op
\]
preserving finite colimits and limits.
\end{thm}

The map is uniquely determined by the following: For all $f\in R$ and $C\in \RR_{>0}$ it is given by
\begin{align*}
\{\lvert f\rvert \le C\} &\mapsto R\otimes_{\CC^{\mathrm{gas}}} \O(C\cdot \DD) /(T-f), \\
\{\lvert f\rvert \ge C\} &\mapsto R\otimes_{\CC^{\mathrm{gas}}} \O\bigl(\{\lvert M\rvert \ge C\}\text{ merom. at $\infty$}\bigr)/(T-f).
\end{align*}

Explicitly, for any closed subset $K\subset \cat M_B(R)$ we get an idempotent $R$-algebra $\O(K)$.

\begin{ex}
Let $R = \O(\DD^n)$. Then 
\[
\O(K) = \varinjlim_{\substack{U\supseteq K\\ \text{open}}} \R\Gamma(U; \O^{\mathrm{hol}}).
\]
\end{ex}

\begin{rem}
In general, $\O(K)$ can live in positive and negative degrees. In practice it lives in degrees $\le 0$.

Moreover, in general, $\set{K}{\O(K) \in D_{\ge0}(\CC^{\mathrm{gas}})}$ is closed under intersections and generates the topology. For any such $K$, we have $\O(K) \in \CAlg(D_{\ge0}(\CC^{\mathrm{gas}}))^{\mathrm{bded}}$ and
\[
\begin{tikzcd}
\cat M_B(\O(K)) \ar[d,leftrightarrow,"\simeq"'] \ar[r] & \cat M_B(R) \\
K \ar[ur,hook],
\end{tikzcd}
\]
and for $K'\subseteq K$ the idempotent algebras agree. (This is analogous to distinguished opens in algebraic geometry.)
\end{rem}

\begin{ex}
If $R =\O(\DD^n)$, then $K \subseteq \DD^n$ satisfies $\O(K) \in D_{\ge0}(\CC^{\mathrm{gas}})$ if and only if $K$ is holomorphic (?) convex (compact Stein).
\end{ex}

Recall (Lurie), if $X$ is locally compact Hausdorff and $\cat C \in \Pr^L_{\mathrm{st}}$, then
\[
\Shv(X;\cat C) \xleftrightarrow{\simeq} \Shv_K(X;\cat C),
\]
where the right hand side is the category of presheaves on compact subsets such that 
\begin{enumerate*}[(1)]
\item it satisfies the sheaf condition for finite covers and 
\item $\sheaf F(K) = \varinjlim_{K\Subset K'} \sheaf F(K')$.
\end{enumerate*}

For $X = \cat M_B(R)$ one has the same if one only restricts to $K$ such that $\O(K) \in D_{\ge0}(\CC^{\mathrm{gas}})$.

\begin{cor}
\begin{enumerate}[(a)]
\item We get a structure sheaf $\O \in \Shv(\cat M_B(R); \CAlg(D(\CC^{\mathrm{gas}})))$. (For $R = \O(\DD)$ we get the usual $\O^{\mathrm{hol}}$.)

\item We get a sheaf with values in $\CAlg(\Pr^L)$, given by $K\mapsto D(\O(K))$; this uses idempotency.
\end{enumerate}
\end{cor}

\begin{thm}[Automatic quasicoherence]
The functor
\begin{align*}
D(R) &\to \Mod_{\O}(\Shv(\cat M_B(R); D(\CC^{\mathrm{gas}}))), \\
M &\mapsto \bigl(K\mapsto M\otimes_R\O(K)\bigr)
\end{align*}
is an equivalence.
\end{thm}
\begin{proof}
Fully faithfulness is easy. For essential surjectivity it is enough to hit the generators \enquote{$h_U$}, where $U \subseteq \cat M_B(R)$ is open. These are hit by $\fib(R\to \O(X\setminus U))$.
\end{proof}

We thus get a category $D_{\mathrm{qcoh}}(M)$ for any complex manifold $M$.

\begin{thm}
Let $R$ be bounded. Then we can define a (derived de\,Rham) complex on $\cat M_B(R)$, which is also a sheaf.
\end{thm}

\begin{ex}
If $R = \O(\DD^n)$, then we get back the usual de\,Rham complex.

The key input is that $\CC[T_1,\dotsc, T_n] \injto \O(\DD^n)$ is idempotent.
\end{ex}

Note that this also allows us to define Deligne cohomology.

\begin{warning}
The category $D(\CC^{\mathrm{gas}})$ is not rigid.
\end{warning}

The fix is to pass to a full subcategory which \emph{is} rigid. Recall that, if $\cat C\in \CAlg(\Pr^L)$ is such that $\one$ is compact, then $\cat C$ is rigid if and only if $\cat C$ is generated by \emph{basic nuclear} objects (which is the same as $\omega_1$-compact objects), i.e., objects of the form $\varinjlim (x_0\to x_1\to \dotsb)$, where all transition maps $x_n\to x_{n+1}$ are trace class.\footnote{If $\cat C$ is compactly generated, then $X\in \cat C$ is nuclear if and only if $(\iHom(K,\one)\otimes X)(*)\isoto \Hom(K,X)$ is an isomorphism for all compact $K$.} This is not satisfied for $\cat C = D(\CC^{\mathrm{gas}})$. For example, $P = \CC^{\mathrm{gas}}(\NN\cup\{\infty\}/\infty)$ is compact in $D(\CC^{\mathrm{gas}})$. But it is not basic nuclear, because $\id\colon P\to P$ is not trace class.

In fact, we have $P\subseteq \prod_{\NN} \CC$ consisting of those sequences with \enquote{quasi-exponential decay}. The trace class maps in $D(\CC^{\mathrm{gas}})$ are generated by maps $P\to P$ which are given by a diagonal matrix with quasi-exponential decay.

\begin{defn}
Let $\cat C \in \CAlg(\Pr^L)$ and assume that $\one$ is compact. Let
\[
\Nuc(\cat C) \subseteq \cat C
\]
be the full subcategory generated by the basic nuclear objects.
\end{defn}

\begin{fact}
$\Nuc(\cat C)$ is closed under $\otimes$, and we have $\one\in \Nuc(\cat C)$.
\end{fact}

\begin{question}
Is $\Nuc(\cat C)$ always rigid?
\end{question}

In general, the answer is \emph{no}! But the answer is yes if every trace class map factors as the composite of two trace class maps. This holds for $D(\CC^{\mathrm{gas}})$, i.e., $\Nuc(\CC^{\mathrm{gas}})$ is rigid.

\begin{thm}
\begin{enumerate}[(a)]
\item $\Nuc(\CC^{\mathrm{gas}}) = \Nuc(\langle P\rangle)$.
\item Let $R\in \CAlg(D(\CC^{\mathrm{gas}}))$ such that $R\in \Nuc(\CC^{\mathrm{gas}})$, then
\[
\Nuc(\Mod_R(D(\CC^{\mathrm{gas}}))) = \Mod_R(\Nuc(\CC^{\mathrm{gas}})).
\]

\item $\O^{\mathrm{hol}}(\DD^n) \in \Nuc(\CC^{\mathrm{gas}})$.
\item $\Nuc(\CC^{\mathrm{gas}}) \subseteq D(\CC^{\mathrm{gas}})$ is closed under countable limits.
\end{enumerate}
\end{thm}

Hence, for a complex manifold $M$, we can define 
\[
D_{\mathrm{qcoh}}(M) \supseteq \Nuc(M)
\]
such that $\sheaf F$ is nuclear if and only if $\sheaf F(K)$ is a nuclear $\O(K)$-module for all compact $K$ or, equivalently, $\sheaf F(K)$ is nuclear over $\CC^{\mathrm{gas}}$.

\section{Talk 4: Cohomology theories on complex manifolds (joint with P. Scholze)}

\begin{thm}
If $R\neq 0$ is pointwise bounded, then there exists a $\CC$-algebra morphism $R(*) \to \CC$.
\end{thm}
\begin{proof}
If there were no such algebra morphism, then $R = \O(\cat M_B(R)) = \O(\emptyset) = 0$.

\textbf{Challenge:} Give a direct proof.
\end{proof}

\begin{defn}
We put
\[
\begin{tikzcd}[column sep=small]
\Shv(\Man_{\CC}; \Sp) \ar[d,"\simeq"'] \ar[r,equals] & \set{\sheaf F\colon \Man_{\CC}^{\op} \to \Sp}{\forall M\in \Man_{\CC}, \text{$\sheaf F\big|_{\Open(M)}$ is a sheaf}} \\
\Shv(\O(C)\textnormal{'s};\Sp) \ar[r,equals] & \set{\sheaf F\colon \{\O(C)\textnormal{'s}\} \to \Sp}{\forall C, \text{$\sheaf F\big|_{\Closed(C)}$ is a $\cat K$-sheaf}},
\end{tikzcd}
\]
where the $\O(C)$'s, for $C\subseteq \CC^d$ compact Stein, live in $\CAlg(D_{\ge0}(\CC^{\mathrm{gas}}))$.
\end{defn}

\begin{ex}
\begin{enumerate}[(1)]
\item For all $p\in \ZZ$ we have $\ZZ(p)_{\D}(M) = \R\Gamma(M;\ZZ(p)_{\D})$.

\item $K^{\mathrm{nuc}}$: we have $K^{\mathrm{nuc}}(\O(C)) = K^{\mathrm{cont}}(\Nuc(\O(C)))$.

\item K-theory analog (Karoubi)
\[
\begin{tikzcd}
K^{\mathrm{Del}} \ar[d] \ar[r] \ar[dr,phantom, very near start, "\lrcorner"] & \mathrm{HC}^-(\O(K)/\CC^{\mathrm{gas}}) \ar[d] \\
\ul{KU} \ar[r] & \mathrm{HP}(\O(K)/\CC^{\mathrm{gas}}),
\end{tikzcd}
\]
where the bottom map is induced by $ku \to \mathrm{HP}(\CC/\CC^{\mathrm{gas}}) = \prod_{n\in\ZZ} \CC(2n)$.

There exists a filtration on $K^{\mathrm{Del}}$ with associated graded pieces $\ZZ(p)_{\D}[2p]$, which rationally splits:
\[
K^{\mathrm{Del}}_{\QQ} = \bigoplus_p \QQ(p)_{\D}[2p].
\]
\end{enumerate}
\end{ex}

\begin{thm}
There exists a natural map $K^{\mathrm{nuc}} \to K^{\mathrm{Del}}$.
\end{thm}

\begin{mainconj}[Modified Hodge Conjecture]
The map above is an isomorphism.
\end{mainconj}

\begin{thm}
In any case, $K^{\mathrm{Del}}$ (and all three terms in its definition) is an invariant of $\Nuc(\O(K))$ (as a $\Nuc(\CC^{\mathrm{gas}})$-linear category).
\end{thm}

We obtain proper pushforward, which rather easily implies Riemann--Roch theorems.

\begin{defn}
A sheaf $\sheaf F\in \Shv(\Man_{\CC}; \Sp)$ is \emph{$\DD^0$-invariant} if the pullback map
\[
\sheaf F(M) \isoto \sheaf F(\DD^0\times M)
\]
is an isomorphism for all $M\in \Man_{\CC}$.
\end{defn}

\begin{thm}
\begin{enumerate}[(a)] 
\item For $\sheaf F\in \Shv(\Man_{\CC}; \Sp)$ the following are equivalent:
\begin{enumerate}[(1)]
\item $\sheaf F$ is $\DD^0$-invariant.
\item  $\sheaf F$ (viewed as a $\cat K$-sheaf) is $[0,1]$-invariant (where $[0,1] \subset \CC$).

\item For all $d\ge0$, the map $\sheaf F(\O_{\CC^d, 0}) \isoto \sheaf F(\CC)$ is an isomorphism.
\end{enumerate}

\item The functor $\Shv_{\DD^0}(\Man_{\CC}; \Sp)\isoto \Sp$, $\sheaf F\mapsto \sheaf F(*)$ is an equivalence.

\item The inclusion $\Shv_{\DD^0}(\Man_{\CC};\Sp)\subseteq \Shv(\Man_{\CC}; \Sp)$ has a left and a right adjoint.

The right adjoint is given by $\sheaf F\mapsto (M\mapsto \R\Gamma(M; \sheaf F(*)))$.

The left adjoint is given by $\sheaf F \mapsto \sheaf F^h\coloneqq \varinjlim_{[n]\in \Delta} \sheaf F(\blank \times \Delta^n)$, where $\sheaf F$ is viewed as a $\cat K$-sheaf.
\end{enumerate}
\end{thm}

\begin{ex}
\begin{enumerate}[(i)]
\item Let $M\in \Man_{\CC}$. Then
\[
\SS[h_M]^h = \R\Gamma(\blank; \SS(\Pi_\infty M)).
\]

\item The \enquote{homotopification} of the sheafification $\bigl(\wt{K(\Vect(\blank))}\bigr)^h$ identifies with $\R\Gamma(\blank, ku)$.
\end{enumerate}
\end{ex}

\begin{thm}\label{Clausen:thm}
\begin{enumerate}[(a)]
\item We have
\[
(K^{\mathrm{nuc}})^h = \R\Gamma(\blank; KU).
\]

\item The map $K^{\mathrm{nuc}} \to (K^{\mathrm{nuc}})^h$ is an isomorphism in degrees $\le 0$ (on any $C$).

\item $(\mathrm{HC}^-)^h = \mathrm{HP}$.
\end{enumerate}
\end{thm}

The theorem implies the existence of a natural trace map
\[
\begin{tikzcd}
K^{\mathrm{nuc}} \ar[dr] \ar[drr, bend left,"{\text{$\exists$ trace}}"] \ar[ddr, bend right] &[-1em] \\[-1em]
& K^{\mathrm{Del}} \ar[r] \ar[d] & \mathrm{HC}^- \ar[d] \\
& (K^{\mathrm{nuc}})^h \ar[r] & (\mathrm{HC}^-)^h,
\end{tikzcd}
\]
which uses rigidity of $\Nuc(\CC^{\mathrm{gas}})$.

\begin{cor}
The following are equivalent:
\begin{enumerate}[(a)]
\item The modified Hodge conjecture holds.
\item $\fib(K^{\mathrm{nuc}} \to \mathrm{HC}^-)$ is a $\DD^0$-invariant sheaf.

\item $\dagger$-rigidity: the commutative square
\[
\begin{tikzcd}
K^{\mathrm{nuc}}(\O_{\CC^\dagger,0}) \ar[d] \ar[r] \ar[dr,phantom, very near start, "\lrcorner"] & \mathrm{TC}^-(\O_{\CC^d, 0}/\CC^{\mathrm{gas}}) \ar[d] \\
K^{\mathrm{nuc}}(\CC) \ar[r] & \mathrm{TC}^-(\CC/\CC)
\end{tikzcd}
\]
is a pullback.
\end{enumerate}
\end{cor}

\begin{rem}
Replace $\CC^{\mathrm{gas}}$ by $\QQ_p^{\mathrm{solid}}$. Then $\dagger$-rigidity is true!
\end{rem}

\begin{rem}[Conjecture]
$\mathrm{HH}(\O(C)/\CC^{\mathrm{gas}}) = \mathrm{HH}^{\mathrm{cont}}(\Nuc(\O(C))/\QQ)$.

Again, this is true in the non-archimedean analog (due to Cordova).
\end{rem}

\begin{proof}[Proof of \cref{Clausen:thm}]
\textbf{Step 1:} For (a), use the commutative diagram
\[
\begin{tikzcd}
ku \ar[r,equals] &[-1em]  (K^{\mathrm{vect}})^h \ar[d] \ar[r] & (K^{\mathrm{nuc}})^h \\
& KU \ar[ur,dashed, "\exists"'],
\end{tikzcd}
\]
i.e., $\beta\in \pi_2(ku)$ is invertible in $(K^{\mathrm{nuc}})^h$.

For the proof, use GAGA: If $X$ is a smooth proper scheme over $\CC$, then
\[
D_{\mathrm{qc}}(X) \otimes_{D(\CC(*))} \Nuc(\CC^{\mathrm{gas}}) \isoto \Nuc_{\mathrm{qc}}(X^{\mathrm{an}}).
\]
Apply this to $X = \PP^1$. Then $\Nuc(\PP^1) = \langle \Nuc(\CC^{\mathrm{gas}}), \Nuc(\CC^{\mathrm{gas}})\rangle$, which implies
\[
K^{\mathrm{nuc}}(\PP^1\times C) \simeq K^{\mathrm{nuc}}(C) \oplus K^{\mathrm{nuc}}(C).
\]
Applying $(\blank)^h$, we deduce an inverse for $\beta$.
\medskip

\textbf{Step~2:} The sheaf $\pi_0 K^{\mathrm{nuc}}(\O(\blank))$ is $[0,1]$-invariant.
\begin{proof}
The map $K(\O(C)(*)) \to K^{\mathrm{nuc}}(C) = K^{\mathrm{cont}}(\Nuc(\O(C)))$ is an isomorphism on $\pi_0$ (and only on $\pi_0$!). \enquote{The proof is tricky but sort of straightforward.}

By Grauert--Oka, $K_0(\Vect(\O(C)))$ is homotopy invariant.
\end{proof}

\textbf{Step~3:} The sheaf $\tau_{\le 0}K^{\mathrm{nuc}}$ is $[0,1]$-invariant.

For the proof, one uses Bass delooping to compute 
\[
K^{\mathrm{nuc}}_{-1}(C) = \coker\bigl(K_0^{\mathrm{nuc}}(\DD_+\times C) \oplus K_0^{\mathrm{nuc}}(\DD_-\times C) \to K_0^{\mathrm{nuc}}(S^1\times C) \bigr)
\]
and then do descending induction.
\medskip

\textbf{Step 4:} The map $K^{\mathrm{nuc}} \to (K^{\mathrm{nuc}})^h$ is an isomorphism on $\tau_{\le0}$.

We have $(K^{\mathrm{nuc}})^h = \varinjlim K^{\mathrm{nuc}}(\blank\times \Delta^n)$, so the claim follows immediately from Step~3.
\medskip

\textbf{Step 5:} We have $(K^{\mathrm{nuc}})^h = \R\Gamma(\blank; (K^{\mathrm{nuc}})^h(*))$, which is a $KU$-module by Step~1, hence is 2-periodic. But on the other hand, the map $K^{\mathrm{nuc}} \to (K^{\mathrm{nuc}})^h$ is an isomorphism on $\pi_0$ and $\pi_{-1}$.

It thus suffices to show 
\begin{align*}
\pi_0K^{\mathrm{nuc}}(\CC) &= \ZZ, \\
\pi_{-1}K^{\mathrm{nuc}}(\CC) &= 0.
\end{align*}
\end{proof}





\end{document}










