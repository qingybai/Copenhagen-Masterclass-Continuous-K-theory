\documentclass[draft]{amsart}
\usepackage[utf8]{inputenc}
\usepackage[T1]{fontenc}
\usepackage{amssymb,amsthm}
\usepackage{mathtools}
\usepackage{tikz-cd}
\usepackage{stmaryrd} % \mapsfrom
\usepackage[shortlabels]{enumitem}
\usepackage{geometry}
\usepackage[final,kerning,spacing]{microtype}\frenchspacing
\usepackage{csquotes}
\usepackage[final]{hyperref}
\usepackage[capitalize]{cleveref}
\usepackage[inline]{showlabels}

\newcommand{\todo}[1]{\textcolor{red}{TODO: #1}}

\newcommand{\NN}{\mathbb{N}}
\newcommand{\ZZ}{\mathbb{Z}}
\newcommand{\QQ}{\mathbb{Q}}
\newcommand{\RR}{\mathbb{R}}
\newcommand{\CC}{\mathbb{C}}
\newcommand{\FF}{\mathbb{F}}
\renewcommand{\AA}{\mathbb{A}}
\newcommand{\GG}{\mathbb{G}}
\newcommand{\PP}{\mathbb{P}}

\renewcommand{\H}{\mathrm{H}}
\newcommand{\D}{\mathrm{D}}
\newcommand{\R}{\mathrm{R}}
\renewcommand{\O}{\mathcal{O}}

\newcommand{\set}[2]{\left\{#1\,\middle|\,#2\right\}}
\newcommand{\ol}[1]{\overline{#1}}
\newcommand{\ul}[1]{\underline{#1}}
\newcommand{\wh}[1]{\widehat{#1}}
\newcommand{\wt}[1]{\widetilde{#1}}
\newcommand{\cat}[1]{\mathcal{#1}}
\newcommand{\sheaf}[1]{\mathcal{#1}}

\renewcommand{\setminus}{\smallsetminus}
\renewcommand{\emptyset}{\vanothing}
\newcommand{\ssubset}{\Subset} 
\newcommand{\ssupset}{\Supset} 
\newcommand{\op}{\mathrm{op}}

\newcommand{\too}{\longrightarrow}
\newcommand{\xto}[1]{\mathbin{\xrightarrow{#1}}}
\newcommand{\isoto}{\mathbin{\xrightarrow{\sim}}}
\newcommand{\injto}{\mathbin{\hookrightarrow}}
\newcommand{\injtoo}{\mathbin{\longhookrightarrow}}
\newcommand{\epito}{\mathbin{\to\kern-.8em\to}}
\newcommand{\epitoo}{\mathbin{\too\kern-.8em\to}}

\renewcommand{\Pr}{\mathrm{Pr}}
\newcommand{\Stab}{\mathrm{Stab}}

\newcommand{\blank}{-} % todo

\DeclareMathOperator{\Spec}{Spec}
\DeclareMathOperator{\Hom}{Hom}
\DeclareMathOperator{\fib}{fib}
\DeclareMathOperator{\End}{End}
\DeclareMathOperator{\Spa}{Spa}
\DeclareMathOperator{\dlog}{dlog}
\DeclareMathOperator{\SingularSupport}{SS}
\DeclareMathOperator{\Ind}{Ind}
\DeclareMathOperator{\Map}{Map}
\DeclareMathOperator{\Open}{Open}
\DeclareMathOperator{\Seminorm}{Seminorm}
\DeclareMathOperator{\Shv}{Shv}
\DeclareMathOperator{\Nuc}{Nuc}
\DeclareMathOperator{\Cat}{Cat}
\DeclareMathOperator{\CompAss}{CompAss}
\DeclareMathOperator{\Sp}{Sp}
\DeclareMathOperator{\Fun}{Fun}
\DeclareMathOperator{\Mod}{Mod}
\DeclareMathOperator{\Perf}{Perf}
\DeclareMathOperator{\CompHaus}{CompHaus}
\DeclareMathOperator{\Spf}{Spf}
\DeclareMathOperator{\PSh}{PSh}
\DeclareMathOperator{\Corr}{Corr}
\DeclareMathOperator{\Pro}{Pro}
\DeclareMathOperator{\Image}{Im}
\DeclareMathOperator{\Tate}{Tate}
\DeclareMathOperator{\Ext}{Ext}
\DeclareMathOperator{\Vect}{Vect}
\DeclareMathOperator{\Spv}{Spv}
\DeclareMathOperator{\Calk}{Calk}
\DeclareMathOperator{\QCoh}{QCoh}
\DeclareMathOperator{\GL}{GL}
\DeclareMathOperator*{\indinjlim}{\text{``}\varinjlim\text{''}} % todo
\DeclareMathOperator*{\indprojlim}{\text{``}\varprojlim\text{''}} % todo
\DeclareMathOperator*{\colim}{colim}

\newtheorem{thm}{Theorem}[section]
\newtheorem{prop}[thm]{Proposition}
\newtheorem{lem}[thm]{Lemma}
\newtheorem{cor}[thm]{Corollary}

\theoremstyle{definition}
\newtheorem{defn}[thm]{Definition}
\newtheorem{rem}[thm]{Remark}
\newtheorem{fact}[thm]{Fact}
\newtheorem{ex}[thm]{Example}
\newtheorem{exercise}[thm]{Exercise}
\newtheorem{warning}[thm]{Warning}
\newtheorem*{claim}{Claim}
\newtheorem*{notation}{Notation}

\begin{document}
\title{Three persectives on Deligne cohomology}
\author{Dustin Clausen}
\maketitle
\tableofcontents


\section{Talk 1}

Deligne cohomology is a cohomology theory for complex manifolds which refines the usual singular/sheaf cohomology $\H^*(M;\ZZ)$ by including some differential form data.

\subsection{Motivation}
Let $V$ be a holomorphic vector bundle over a complex manifold $M$. Then we get a complex topological vector bundle on the topological space $M$, hence a Chern class $c_p(V) \in \H^{2p}(M;\ZZ)$.

We want refined coefficients $\ZZ(p)_{\D}$ which maps to $\ZZ$ such that $c_p(V) \in \H^{2p}(M;\ZZ)$ functorially (with respect to pullback of vector bundles) lifts to $\H^{2p}(M; \ZZ(p)_{\D})$.

\begin{rem}
Why do the coefficients depend on $p$? Note that $c_1(\sheaf L) \in \H^{2}(M;\ZZ)$, where $\ZZ$ should be identified with $\H_1(\CC^\times; \ZZ) \cong 2\pi i\ZZ \subseteq \CC$.

Similarly, we should have $c_p(V) \in \H^{2p}(M; (2\pi i)^{p}\ZZ)$.
\end{rem}

To define $\ZZ(p)_{\D}$, let us look at first Chern classes:
\[
c_1(V) = c_1(\det(V)).
\]
One description is the following: The short exact sequence
\[
0\to 2\pi i\ZZ \to \O \xrightarrow{\exp} \O^\times \to 1.
\]
gives a map $\H^1(M;\O^\times) \to \H^2(M;2\pi i\ZZ)$. This suggests setting
\[
\ZZ(1)_{\D} = \O^\times[-1].
\]

\textbf{Reinterpretation:} We have a homotopy pullback (because the cofibers on both horizontal maps identify with $\O$):
\[
\begin{tikzcd}
\O^\times{[-1]} \ar[r,"\partial"] \ar[d,"\dlog(f) = \frac{\mathrm{d} f}{f}"'] & 2\pi i\ZZ \ar[d] \\
{[0\to \Omega^1\to \Omega^2\to\dotsb]} \ar[r] & {[\Omega^0 \to \Omega^1 \to \dotsb]}
\end{tikzcd}
\]
where $\Omega_{\mathrm{dR}} \coloneqq (\Omega^\bullet,d)$ is the holomorphic de\,Rham complex.

\begin{defn}
For $p\in \ZZ$, $p\ge0$, define
\[
\begin{tikzcd}
\ZZ(p)_{\D} \ar[d] \ar[dr,phantom, very near start, "\lrcorner"] \ar[r] & (2\pi i)^p\ZZ \ar[d] \\
F^p\Omega_{\mathrm{dR}} = {[0\to \dotsb\to 0\to \Omega^p\to \Omega^{p+1}\to \dotsb]} \ar[r] & {[\Omega^0\to \Omega^1\to\dotsb]} = \Omega_{\mathrm{dR}},
\end{tikzcd}
\]
where on the left $\Omega^p$ sits in homological degree $-p$.
\end{defn}

\begin{ex}
\begin{enumerate}[(i)]
\item $\ZZ(0)_{\D} = \ZZ$;
\item $\ZZ(1)_{\D} = \O^\times[-1]$, which implies that there is a tautological Chern class $c_1(\sheaf L) \in \H^2(M; \ZZ(1)_{\D})$.

\item $\bigoplus_p \ZZ(p)_{\D}$ is a graded commutative ring, i.e., there is a map $\ZZ(p)_\D \otimes \ZZ(q)_{\D} \to \ZZ(p+q)_{\D}$.

\item Pullback functoriality: for each $M\to N$ there is a map $\H^*(N; \ZZ(p)_{\D}) \to \H^*(M;\ZZ(p)_{\D})$.

\item Projective bundle formula: Let $V \to M$ be a vector bundle of dimension $d$ and consider its projectivization $\PP(V) \to M$. Then
\[
\bigoplus_p\R\Gamma(\PP(V); \ZZ(p)_{\D})
\]
is graded free of rank $d$ over $\bigoplus_p \R\Gamma(M;\ZZ(p)_{\D})$ on $1, c_1(\O(1)), \dotsc, c_1(\O(1))^{d-1}$.

By Grothendieck, we can expand $c_1(\O(1))^d$ in terms of previous powers, and the coefficients define the higher Chern classes $c_p(V) \in \H^{2p}(M; \ZZ(p)_{\D})$.
\end{enumerate}
\end{ex}

\begin{ex}
If $p\le 0$, then $\ZZ(p)_{\D} = (2\pi i)^p\ZZ$, which is \enquote{purely topological}.

If $p>\dim M$, then $\ZZ(p)_{\D} = \CC/(2\pi i)^p \ZZ[-1]$.
\end{ex}

\begin{rem}
Let $M$ be Stein (i.e., a closed submanifold $M\injto \CC^N$). Then $\Omega^i$ is acyclic and hence $\H^p(F^p\Omega_{\mathrm{dR}}) = \Omega^p_{\mathrm{cl}}$ is the space of holomorphic closed $p$-forms (these are huge vector spaces!).
\end{rem}

If $M$ is compact, then $\dim_{\CC} (\bigoplus_{i,p} \H^i(M; \Omega^p)) < \infty$ and hence the Deligne cohomlogy groups are always built out of $\ZZ$'s and $\CC$'s by extensions and quotients.

If $M$ is compact K\"ahler (e.g., a smooth projective variety over $\CC$), then the map
\[
\H^*(M; F^p\Omega_{\mathrm{dR}}) \to \H^*(M; \Omega_{\mathrm{dR}}) \simeq \H^*(M;\CC)
\]
is \emph{injective}. The image is $F^p\H^*(M;\CC)$ in the Hodge filtration. Thus, we have a short exact sequence
\[
0\to \frac{\H^{i-1}(M;\CC)}{F^p\H^{i-1}(M;\CC) + \H^{i-1}(M; (2\pi i)^{p-1}\QQ)} \to 
\H^i(M; \QQ(p)_{\D}) \to F^p\H^i(M;\CC) \cap \H^i(M; (2\pi i)^p\QQ) \to 0,
\]
which we view as an extension of something \enquote{discrete} by something \enquote{continuous}.

When $i = 2p$, the left hand side is $J^p(M)_{\QQ}$, that is, Griffith's intermediate Jacobian; for $p=1$ this is the usual Jacobian.

\subsection{Applications} 
\begin{enumerate}[(i)]
\item One application, the intermediate Jacobians $J^p(M)$, were already mentioned.

\item Secondary characteristic classes of flat bundles
\[
\begin{tikzcd}
c_p(V) \ar[r,phantom,"\in"] &[-2em]  \H^{2p-1}(B\GL_n(\CC)^{\delta}, \CC/(2\pi i)^p\ZZ), \ar[d,"\partial"] & p>0 \\
& \H^{2p}(B\GL_n(\CC)^{\delta}, (2\pi i)^p\ZZ),
\end{tikzcd}
\]
also called the Chern--Simons invariants.

\item \textbf{Arithmetic:} Let $\cat{X} \to \Spec(\ZZ)$ be a regular proper scheme over $\ZZ$. Then $\cat X$ should not be thought of as compact, because $\Spec(\ZZ)$ is not compact ($\Spec(\ZZ)$ corresponds to the affine line $\Spec(\FF_p[T]) = \AA^1_{\FF_p}$. 

There is a small neighborhood around $\infty \in \PP^1_{\FF_p}$ corresponding to $\FF_p[T] \to \FF_p((T^{-1}))$ (which which should be thought of as corresponding to the inclusion $\{0\} \to \RR$).

On $\cat X$, we should consider not just cohomology, but \enquote{compactly supported cohomology}, namely
\[
\fib\bigl(\R\Gamma(\cat X) \to \R\Gamma(\cat X(\CC)/C_2)\bigr).
\]
The idea of Arakelov theory is the following: If \enquote{cohomology} means motivic cohomology, then $\R\Gamma(\cat X(\CC))$ should be Deligne cohomology.
\end{enumerate}

\begin{ex}
Consider the motivic cohomology
\[
\H^i_{\mathrm{M}}(\Spec(\ZZ);\QQ(p)) \to \H^i(*_{\CC}; \QQ(p)_{\D})^{C_2}.
\]
In weights $p>1$, (by Borel) the left hand side is non-zero if and only if $i=1$ and $p$ is odd, in which case it is a one-dimensional $\QQ$-vector space. The right hand side is non-zero if and only if $i=1$ and $p$ is odd, in which case it is isomorphic to $\CC/(2\pi i)^p\QQ \xrightarrow{\mathrm{Re}} \RR$. The image of the induced map $\QQ \to \RR$ can be identified with $\pi^? \zeta(p)\QQ$, where $\zeta$ is the Riemann $\zeta$-function.
\end{ex}

\subsection{Goals of this lecture series} 
\begin{enumerate}[(a)]
\item We want to make precise the idea that Deligne cohomology is the analog of motivic cohomology for complex manifolds.

\item A second goal is to understand the Hodge conjecture: For a complex smooth projective manifold $M$ we have:
\[
\begin{tikzcd}
K_0(\Vect(M))_{\QQ} \ar[dr,twoheadrightarrow] \ar[r,"\mathrm{ch}"] & \bigoplus_p \H^{2p}(M; \QQ(p)) \ar[d,twoheadrightarrow] \\
& \bigoplus_p \mathrm{Hdg}^p(M)_{\QQ}.
\end{tikzcd}
\]

We want to modify $K_0(\Vect(M))$ using continuous K-theory to get a theory where it is reasonable to conjecture that $K_0(\Nuc(M))_\QQ \isoto \bigoplus_p \H^{2p}(M;\QQ(p))$.

\item A third goal is to make Riemann--Roch more transparent.
\end{enumerate}





\end{document}











