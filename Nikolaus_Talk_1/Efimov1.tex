\documentclass[draft]{amsart}
\usepackage[utf8]{inputenc}
\usepackage[T1]{fontenc}
\usepackage{amssymb,amsthm}
\usepackage{mathtools}
\usepackage{tikz-cd}
\usepackage{stmaryrd} % \mapsfrom
\usepackage[shortlabels]{enumitem}
\usepackage{geometry}
\usepackage[final,kerning,spacing]{microtype}\frenchspacing
\usepackage{csquotes}
\usepackage[final]{hyperref}
\usepackage[capitalize]{cleveref}
\usepackage[inline]{showlabels}

\newcommand{\todo}[1]{\textcolor{red}{TODO: #1}}

\newcommand{\NN}{\mathbb{N}}
\newcommand{\ZZ}{\mathbb{Z}}
\newcommand{\QQ}{\mathbb{Q}}
\newcommand{\RR}{\mathbb{R}}
\newcommand{\CC}{\mathbb{C}}
\newcommand{\FF}{\mathbb{F}}
\renewcommand{\AA}{\mathbb{A}}
\renewcommand{\SS}{\mathbb{S}}
\newcommand{\GG}{\mathbb{G}}
\newcommand{\PP}{\mathbb{P}}

\renewcommand{\H}{\mathrm{H}}
\newcommand{\D}{\mathrm{D}}
\newcommand{\R}{\mathrm{R}}
\renewcommand{\O}{\mathcal{O}}

\newcommand{\set}[2]{\left\{#1\,\middle|\,#2\right\}}
\newcommand{\ol}[1]{\overline{#1}}
\newcommand{\ul}[1]{\underline{#1}}
\newcommand{\wh}[1]{\widehat{#1}}
\newcommand{\wt}[1]{\widetilde{#1}}
\newcommand{\cat}[1]{\mathcal{#1}}
\newcommand{\sheaf}[1]{\mathcal{#1}}

\renewcommand{\setminus}{\smallsetminus}
\renewcommand{\emptyset}{\varnothing}
\newcommand{\ssubset}{\Subset} 
\newcommand{\ssupset}{\Supset} 
\newcommand{\op}{\mathrm{op}}

\newcommand{\from}{\leftarrow}
\newcommand{\too}{\longrightarrow}
\newcommand{\xto}[1]{\mathbin{\xrightarrow{#1}}}
\newcommand{\isoto}{\mathbin{\xrightarrow{\sim}}}
\newcommand{\injto}{\mathbin{\hookrightarrow}}
\newcommand{\injtoo}{\mathbin{\longhookrightarrow}}
\newcommand{\epito}{\mathbin{\to\kern-.8em\to}}
\newcommand{\epitoo}{\mathbin{\too\kern-.8em\to}}

\renewcommand{\Pr}{\mathrm{Pr}}
\newcommand{\Stab}{\mathrm{Stab}}

\newcommand{\blank}{-} % todo

\DeclareMathOperator{\Spec}{Spec}
\DeclareMathOperator{\Hom}{Hom}
\DeclareMathOperator{\fib}{fib}
\DeclareMathOperator{\Anima}{Ani}
\DeclareMathOperator{\End}{End}
\DeclareMathOperator{\Spa}{Spa}
\DeclareMathOperator{\dlog}{dlog}
\DeclareMathOperator{\SingularSupport}{SS}
\DeclareMathOperator{\Ind}{Ind}
\DeclareMathOperator{\Map}{Map}
\DeclareMathOperator{\Open}{Open}
\DeclareMathOperator{\Seminorm}{Seminorm}
\DeclareMathOperator{\Shv}{Shv}
\DeclareMathOperator{\Nuc}{Nuc}
\DeclareMathOperator{\Cat}{Cat}
\DeclareMathOperator{\CompAss}{CompAss}
\DeclareMathOperator{\Sp}{Sp}
\DeclareMathOperator{\Fun}{Fun}
\DeclareMathOperator{\Mod}{Mod}
\DeclareMathOperator{\Perf}{Perf}
\DeclareMathOperator{\CompHaus}{CompHaus}
\DeclareMathOperator{\Spf}{Spf}
\DeclareMathOperator{\PSh}{PSh}
\DeclareMathOperator{\Corr}{Corr}
\DeclareMathOperator{\Pro}{Pro}
\DeclareMathOperator{\Image}{Im}
\DeclareMathOperator{\Tate}{Tate}
\DeclareMathOperator{\Ext}{Ext}
\DeclareMathOperator{\Vect}{Vect}
\DeclareMathOperator{\Spv}{Spv}
\DeclareMathOperator{\Calk}{Calk}
\DeclareMathOperator{\id}{id}
\DeclareMathOperator{\Ker}{Ker}
\DeclareMathOperator{\QCoh}{QCoh}
\DeclareMathOperator{\Cone}{Cone}
\DeclareMathOperator{\GL}{GL}
\DeclareMathOperator*{\indinjlim}{\text{``}\varinjlim\text{''}} % todo
\DeclareMathOperator*{\indprojlim}{\text{``}\varprojlim\text{''}} % todo
\DeclareMathOperator*{\colim}{colim}

\newtheorem{thm}{Theorem}[section]
\newtheorem{prop}[thm]{Proposition}
\newtheorem{lem}[thm]{Lemma}
\newtheorem{cor}[thm]{Corollary}

\theoremstyle{definition}
\newtheorem{defn}[thm]{Definition}
\newtheorem{rem}[thm]{Remark}
\newtheorem{applications}[thm]{Applications}
\newtheorem{fact}[thm]{Fact}
\newtheorem{ex}[thm]{Example}
\newtheorem{exercise}[thm]{Exercise}
\newtheorem{warning}[thm]{Warning}
\newtheorem*{claim}{Claim}
\newtheorem*{notation}{Notation}

\begin{document}
\title{Dualizable Categories and Localizing Motives}
\author{Alexander Efimov}
\maketitle
\tableofcontents

\section{Talk 1}
Consider a qcqs scheme $X$. There are two categories associated with $X$: the category $\Perf(X)= D_{\mathrm{qc}}(X)^{\omega}$ of perfect complexes and $D_{\mathrm{qc}}(X) = \Ind(\Perf(X))$.

There are two invariants we might consider:
\begin{align*}
K(X) &= K(\Perf(X)) \\
\H\H(X) &= \H\H(\Perf(X)), \\
K(D_{\mathrm{qc}}(X)) &= 0.
\end{align*}
The idea is to understand $K(X)$ in terms of $D_{\mathrm{qc}}(X)$.

\subsection{Compactly assembled categories}
Compactly assembled categories were first developed by Lurie, Joyal, Johnstone etc.

If $\cat C = \Ind(\cat A)$, then for all $x\in \cat C$ we have an ind-system $(x_i)_i$ with $x_i\in \cat C^\omega$ such that $x = \varinjlim\limits_i x_i$. Observe that the assignment $x\mapsto \indinjlim_i x_i$ is gives a well-defined functor 
\[
\wh{Y}\colon \cat C \to \Ind(\cat C),
\]
which is left adjoint to $\colim\colon \Ind(\cat C)\to \cat C$.

\begin{defn}
Let $\cat C$ be an accessible $\infty$-category with filtered colimits. Then $\cat C$ is \emph{compactly assembled} if there exists $\wh{Y}\colon \cat C\to \Ind(\cat C)$ which is left adjoint to $\colim$.
\end{defn}

\begin{ex}
\begin{enumerate}[(1)]
\item $\Ind(\cat A)$ is compactly assembled.
\item $(\RR\cup \{+\infty\}, \le)$ is compactly assembled with $\wh{Y}(a) = \indinjlim_{b<a}b$.

\item Let $X$ be a locally compact Hausdorff space. Then $\Open(X)$ is compactly assembled with $\wh{Y}(U) = \indinjlim_{V\ssubset U} V$. Also, $\cat K(X)^{\op}$ is compactly assembled with $\wh{Y}(Z) = \indinjlim_{Z'\ssupset Z} Z'$.

\item \textbf{Exercise:} Let $\Seminorm_1$ be the category of $\RR$-vector spaces with a seminorm $\lVert\cdot\rVert$ and contractible maps (i.e., $\lVert f(x)\rVert \le \lVert x\rVert$). If $\dim V < \infty$ and $(V,\lVert\cdot\rVert)$ is normed, then
\[
\wh{Y}(V,\lVert\cdot\rVert) = \indinjlim_{c>1} (V, c \lVert\cdot\rVert)
\]
and $\wh{Y}(\RR, 0) = \indinjlim_{\varepsilon>0} (\RR,\varepsilon \lvert\cdot\rvert)$.

\item $\Shv(X,\ul{\cat C})$ is compactly assembled, where $X$ is a locally compact Hausdorff space and $\ul{\cat C}$ is a presheaf of dualizable categories.

\item The categories $\Nuc(R_{\wh{I}})$ and $\wh{\Nuc}(R_{\wh{I}})$ of nuclear modules are compactly assembled, and $\Nuc(R_{\wh{I}})^{\omega} = \Perf(R_{\wh{I}})$.

\item Let $R$ be an associative ring and $J\subseteq R$ an ideal which is flat as a right module and satisfying $J^2 = J$. Put
\[
\Mod_a(R) = \Mod(R)/\Mod(R/J)
\]
Then $D(\Mod_a(R))$ is compactly assembled.
\end{enumerate}
\end{ex}

We introduce the following notation:
\begin{itemize}
\item $\Cat^{\mathrm{idem}}$ is the category of small idempotent complete categories,

\item $\Cat^{\mathrm{perf}}$ is the category of small idempotent complete small stable categories , and 

\item $\CompAss$ is the category of compactly assembled categories, where the $1$-morphisms are the strongly continuous functors, i.e., $F\colon \cat C\to \cat D$ such that $F$ commutes with filtered colimits and the diagram
\[
\begin{tikzcd}
\cat C \ar[d,"\wh{Y}"'] \ar[r,"F"] & \cat D \ar[d,"\wh{Y}"] \\
\Ind(\cat C)\ar[r,"\Ind(F)"'] & \Ind(\cat D).
\end{tikzcd}
\]

\item Denote $\Cat^{\mathrm{dual}}_{\mathrm{st}} \subseteq \CompAss$ the subcaetgory of compactly assembled stable categories with strongly continuous exact functors.
Note that both $\Cat^{\mathrm{dual}}_{\mathrm{st}}$ and $\CompAss$ are cocomplete.
\end{itemize}

\begin{prop}[Lurie]
\begin{enumerate}[(1)]
\item $\Cat^{\mathrm{idem}}$ is generated under colimits by $[1]$, which is compact.

\item $\Cat^{\mathrm{perf}}$ is generated under colimits by $\Sp^{\omega}$, which is compact.
\end{enumerate}
\end{prop}
\begin{proof}[Sketch]
Let $F\colon \cat C\to \cat D$ such that $\Fun([1],\cat C) \isoto \Fun([1], \cat D)$. Then $F$ is an equivalence.
\end{proof}

\begin{thm}[Urysohn's lemma]\label{Efimov:Urysohn}
\begin{enumerate}[(1)]
\item $\CompAss$ is generated under colimits by $\RR\cup \{\infty\}$, which is $\omega_1$-compact.

\item $\Cat^{\mathrm{dual}}_{\mathrm{st}}$ is generated under colimits by $\Shv_{\RR\times \RR_{\ge0}}(\RR;\Sp)$.
\end{enumerate}
In particular, $\CompAss$ and $\Cat^{\mathrm{dual}}_{\mathrm{st}}$ are $\omega_1$-presentable.\footnote{The last statement ist due to Ramzi.}
\end{thm}

\begin{rem}
The usual Urysohn's lemma for compact Hausdorff spaces says that $\CompHaus^\op$ is generated under colimits by $[0,1]$, which is $\omega_1$-compact.
\end{rem}

\begin{fact}
We have an equivalence
$\CompAss \isoto \Cat^{\mathrm{dual}}$, where $\Cat^{\mathrm{dual}}$ is the category of dualizable objects in $\Pr^L$ whose $1$-morphisms are those functors $F\colon \cat C\to \cat D$ for which the right adjoint $F^R$ is colimit-preserving. The equivalence takes
\begin{align*}
\Ind(\cat A) &\leftrightarrow \PSh(\cat A; \Anima), \\
\cat C &\mapsto \set{F\colon \cat C^\op \to \Anima}{\begin{array}{l}\text{$F$ commutes with} \\ \text{cofiltered limits}\end{array}}, \\
\set{G\colon \cat D^\vee \to \Anima}{\begin{array}{l}\text{$G$ is colimit preserving} \\ \text{and left exact}\end{array}} &\mapsfrom \cat D.
\end{align*}
\end{fact}

Why is $\CompAss$ generated by $\RR\cup \{\infty\}$? It suffices to show that if $F\colon \cat C\to \cat D$ is strongly continuous such that
\[
\Fun^{\mathrm{str.cont.}}(\RR\cup \{\infty\}, \cat C)^{\simeq} \isoto \Fun^{\mathrm{str.cont.}}(\RR\cup \{\infty\}, \cat D)^{\simeq},
\]
then $F$ is an equivalence.

\begin{prop}
Any compactly assembled category is $\omega_1$-accessible.
\end{prop}
\begin{proof}[Sketch]
We have $\wh{Y}(x) = \indinjlim\limits_{i\in I} x_i$, where $I$ is a directed poset. Then $x\simeq \varinjlim\limits_{f\colon \NN\to I} \varinjlim\limits_{n} x_{f(n)}$, where $x_{f(n)}$ is $\omega_1$-compact.
\end{proof}

\begin{defn}
In a compactly assembled category, a map $f\colon x\to y$ is compact if it factors as
\[
\begin{tikzcd}
Y(x)\ar[r,"Y(f)"] \ar[dr] & Y(y) \\
& \wh{Y}(y)\ar[u].
\end{tikzcd}
\]
\end{defn}

\begin{lem}
If $\cat C$ is compactly assembled, and $f\colon x\to y$ in $\cat C$ is compact, then $f = g\circ h$ such that $g$, $h$ are compact.
\end{lem}
\begin{proof}[Sketch]
Write $\wh{Y}(y)= \indinjlim\indinjlim_i y_i$. Then $\wh{Y}(y) = \varinjlim\limits_i \wh{Y}(y_i)$, so we have a factorization 
\[
\begin{tikzcd}
x \ar[rr,"f"] \ar[dr,"g"'] & & y \\
& y_i \ar[ur,"h"'],
\end{tikzcd}
\]
where $g$ and $h$ are compact.
\end{proof}

\begin{lem}
The functor
\begin{align*}
\Fun^{\mathrm{str.cont.}}(\RR\cup \{\infty\}, \cat C) &\to \cat C^{\omega_1}, \\
G &\mapsto G(\infty)
\end{align*}
is essentially surjective.
\end{lem}
\begin{proof}
Let $x \in \cat C^{\omega_1}$. Then $\wh{Y}(x) = \indinjlim (x_0\to x_1\to \dotsb)$ such that each $x_n \to x_{n+1}$ is compact. Define $G_0\colon \ZZ\cup \{\infty\} \to \cat C$ by
\[
G(m) = \begin{cases}
x, & \text{if $h=\infty$,}\\
x_m, & \text{if $m\ge0$,} \\
0, & \text{if $m<0$.}
\end{cases}
\]
Inductively, define compatible $G_n\colon \frac1{2^n}\ZZ\cup\{\infty\} \to \cat C$ such that all transition maps are compact. We obtain $G\colon \ZZ[1/2] \cup\{\infty\} \to \cat C$ and put
\[
G'\colon \ZZ[1/2] \cup\{\infty\} \to \cat C, \qquad G'(r) = \varinjlim\limits_{b<r} G(b).
\]
Take a left Kan extension of $G'$ to $\RR\cup\{\infty\}$ to get a strongly continuous functor $H\colon \RR\cup \{\infty\} \to \cat C$ such that $H(\infty) = x$.
\end{proof}

For $F$ as above, we already know that $F^{\omega_1}\colon \cat C^{\omega_1} \to \cat D^{\omega_1}$ is essentially surjective. In order to finish the proof of \cref{Efimov:Urysohn}.(1), we need to show that it is fully faithful. Consider the pullback diagram
\[
\begin{tikzcd}
\varprojlim\limits_{a>0} \Map_{\cat C}(G(0), H(a)) \ar[r] \ar[d] & \Fun^{\mathrm{str.cont.}} (\RR\cup \{\infty\}, \cat C)^{\simeq} \ar[d] \\
* \ar[r,"{(G,H)}"'] & \Fun^{\mathrm{str.cont.}}(\RR_{\le0}, \cat C)^{\simeq} \times \Fun^{\mathrm{str.cont.}}(\RR_{>0}\cup \{\infty\}, \cat C)^{\simeq}.
\end{tikzcd}
\]
It follows that $\varprojlim\limits_{a>0} \Map_{\cat C}(G(0), H(a)) \isoto \varprojlim\limits_{a>0} \Map_{\cat D}(F(G(0)), F(H(a)))$.

Given strongly continuous functors $G, H\colon \RR\cup \{\infty\} \to \cat C$, we get
\begin{align*}
\Map_{\cat C}(G(\infty), H(\infty)) &= \varprojlim\limits_{a<\infty} \varinjlim\limits_{b<\infty} \Map_{\cat C}(G(a), H(b)) \\
&= \varprojlim\limits_{a<\infty} \varinjlim\limits_{b<\infty} \varprojlim\limits_{c>b} \Map_{\cat C}(G(a), H(c)) \\
&\isoto \varprojlim\limits_{a<\infty} \varinjlim\limits_{b<\infty} \varprojlim\limits_{c>b} \Map_{\cat D}(F(G(a)), F(H(c))) \\
&\isoto \dotsb \isoto \Map_{\cat D}(F(G(\infty)), F(H(\infty))).
\end{align*}
Hence, $F^{\omega_1}$ is fully faithful.
\bigskip

The functor $\iota\colon \Cat^{\mathrm{dual}}_{\mathrm{st}} \to \CompAss$ is conservative and commutes with filtered colimits. It has a left adjoint 
\begin{align*}
\Stab^{\mathrm{cont}} \colon \CompAss &\to \Cat^{\mathrm{dual}}_{\mathrm{st}}, \\
\cat C &\mapsto \set{F\colon \cat C^\op\to \Sp}{\text{$F$ commutes with filtered colmilits}}.
\end{align*}
It follows that $\Stab^{\mathrm{cont}}(\RR\cup\{\infty\}) = \Shv_{\RR\times \RR_{\ge0}}(\RR, \Sp)$ generates $\Cat^{\mathrm{dual}}_{\mathrm{st}}$

\begin{prop}
For a sheaf $\sheaf F \in \Shv(\RR, \Sp)$, the following are equivalent:
\begin{enumerate}[(1)]
\item The singular support (microsupport) $\SingularSupport(\sheaf F)$ is a subset of $\RR\times \RR_{\ge0}$.
\item For all $a<b$, $\sheaf F((-\infty,b)) \isoto \sheaf F((a,b))$.
\end{enumerate}
\end{prop}

\begin{cor}
We have equivalences
\begin{align*}
\Shv_{\RR\times \RR_{\ge0}}(\RR, \Sp) &\simeq \set{F\colon (\RR_{\le})^\op \to \Sp}{\forall a\in \RR, F(a) \isoto \varprojlim\limits_{b<a} F(b)} \\
&\simeq \Stab^{\mathrm{cont}}(\RR\cup\{\infty\}).
\end{align*}
\end{cor}

\section{Talk 2}

\begin{defn}
Let $\cat C$ be a presentable stable category. Then $\cat C$ is called \emph{flat} if $\cat C\otimes \blank \colon \Pr^L_{\mathrm{st}} \to \Pr^L_{\mathrm{st}}$ preserves fully faithful functors.
\end{defn}

\textbf{Question:} Is every $\cat C\in \Pr^L_{\mathrm{st}}$ flat?

\begin{thm}\label{Efimov:flat}
$\cat C$ is flat if and only if $\cat C$ is dualizable.
\end{thm}
\begin{proof}
If $\cat C$ is dualizable, then $\cat C$ is obviously flat, since $\cat C\otimes \blank = \Fun^L(\cat C^\vee, \blank)$. The other direction will be proved below.
\end{proof}

\begin{notation}
Let $\Pr^{\mathrm{acc}}_{\mathrm{st}}$ be the $(\infty,2)$-category of presentable stable categories and accessible exact functors.
\end{notation}

\begin{prop}
\begin{enumerate}[(a)]
\item For any $\cat C \in \Pr^L_{\mathrm{st}}$, there exists a natural oplax 2-functor $\cat C\otimes \blank \colon \Pr^{\mathrm{acc}}_{\mathrm{st}} \to \Pr^{\mathrm{acc}}_{\mathrm{st}}$ (meaning that there are 2-morphisms $\cat C\otimes (F\circ G) \to (\cat C\otimes F)\circ (\cat C\otimes G)$ which need not be invertible) which extends the usual 2-functor $\cat C\otimes \blank \colon \Pr^L_{\mathrm{st}} \to \Pr^L_{\mathrm{st}}$.

\item If $\cat C$ is flat, then $\cat C\otimes \blank$ is an honest 2-functor.
\end{enumerate}
\end{prop}

Assume that $\cat C$ is $\kappa$-presentable and $F\colon \cat D\to \cat E$ is an accessible functor. Then
\[
\cat C\otimes F\colon \cat C\otimes \cat D \simeq \Fun^{\textnormal{$\kappa$-lex}}((\cat C^\kappa)^\op, \cat D) \xrightarrow{F\circ \blank} \Fun((\cat C^\kappa)^\op, \cat E) \to \Fun^{\textnormal{$\kappa$-lex}}((\cat C^\kappa)^\op, \cat E) \cong \cat C\otimes \cat E,
\]
where the second map is given by the left adjoint to the inclusion.
\bigskip

\textbf{Observation:} For presentable stable $\cat D$, $\cat E$, we have an equivalence of categories
\begin{align*}
\Fun^{\mathrm{acc}}(\cat D, \cat E) &\simeq \Corr(\cat D, \cat E) = \set{(\cat T,i_0,i_1)}{\begin{array}{l} \text{$i_0\colon \cat D\to \cat T$ and $i_1\colon \cat E\to \cat T$ are fully faithful} \\
\text{continuous, and $\cat T = \langle i_1(\cat E), i_0(\cat D)\rangle$}
\end{array}} \\
i_1^R\circ i_0 &\mapsfrom (\cat T,i_0,i_1), \\
F &\mapsto \cat E\oplus_{F}\cat D = \{(x\in \cat E, y\in \cat D, x\to F(y))\}
\quad \text{the oplax limit}.
\end{align*}

We obtain a functor
\begin{align*}
\cat C\otimes\blank \colon \Corr(\cat D, \cat E) &\to \Corr(\cat C\otimes \cat D, \cat C\otimes \cat E), \\
(\cat T, i_0, i_1) &\mapsto (\cat C\otimes \cat T, \cat C\otimes i_0, \cat C\otimes i_1).
\end{align*}

We want to understand the composition
\begin{align*}
\Corr(\cat D_1,\cat D_2) \times \Corr(\cat D_0, \cat D_1) &\to \Corr(\cat D_0, \cat D_2), \\
(\cat T_{12}, \cat T_{01}) &\mapsto \cat T_{02} \subseteq \cat T_{012} = \cat T_{01} \sqcup_{\cat D_1} \cat T_{12},
\end{align*}
where $\cat T_{02}$ is generated by the images of $\cat D_0$ and $\cat D_2$.

Note that
\[
\begin{tikzcd}
(\cat C\otimes \cat T_{01}) \bigsqcup_{\cat C\otimes \cat D_1} (\cat C\otimes \cat T_{12}) \ar[r,"\sim"] & \cat C \otimes (\cat T_{01} \sqcup_{\cat D_1} \cat T_{12}) \\
\cat C\otimes \cat T_{02} \ar[r] & (\cat C\otimes \cat T_{01}) \circ (\cat C\otimes \cat T_{12}) \ar[u,hook].
\end{tikzcd}
\]
The bottom map need not be an equivalence in general, because $\cat C\otimes \cat T_{02} \to \cat C\otimes \cat T_{012}$ is not fully faithful in general. But if $\cat C$ is flat, then it is fully faithful, hence we obtain the functor $\cat C\otimes \blank \colon \Pr^{\mathrm{acc}}_{\mathrm{st}} \to \Pr^{\mathrm{acc}}_{\mathrm{st}}$.

We have a fibration $\Corr \to \Delta$ given by
\begin{align*}
\Corr_n &= \bigsqcup_{(\cat D_0,\dotsc, \cat D_n)} \Corr(\cat D_0, \cat D_1,\dotsc,\cat D_n), \\
\Corr(\cat D_0,\cat D_1,\dotsc,\cat D_n) &= \set{(\cat T, i_0,\dotsc,i_n)}{\begin{array}{l}
\text{$i_k\colon \cat D_k\to \cat T$ are fully faithful, continuous} \\
\text{$\cat T = \langle i_n(\cat D_n), \dotsc, i_0(\cat D_0)\rangle$} \\
\text{If $\cat T_{k,k+1} \subset \cat T$ is generated by $\cat D_k, \cat D_{k+1}$, then} \\
\cat T \simeq \cat T_{01} \sqcup_{\cat D_1} \cat T_{12} \sqcup_{\cat D_2} \dotsb \sqcup_{\cat D_n} \cat T_{n-1,n}
\end{array}}.
\end{align*}

We have a functor $\cat C\otimes \blank \colon \Corr(\cat D_0,\cat D_1,\dotsc, \cat D_n) \to \Corr(\cat C\otimes \cat D_0,\dotsc, \cat C\otimes \cat D_n)$ and hence we get a map
\[
\cat C\otimes\blank \colon \Corr \to \Corr
\]
of fibrations over $\Delta$.

\begin{proof}[Proof of \cref{Efimov:flat}]
We now prove that if $\cat C$ is flat, then $\cat C$ is dualizable. Recall that $\cat C$ is dualizable if and only if (AB6) holds in $\cat C$: that is, for all directed posets $J_i$, $i\in I$, and functors $J_i \to \cat C$, $j_i\mapsto x_{j_i}$, then the map
\[
\varinjlim\limits_{(j_i)_i \in \prod_i J_i} \prod_i x_{j_i} \isoto \prod_i \varinjlim\limits_{j_i} x_{j_i}.
\]
We have a commutative square
\begin{equation}\label{Efimov:AB6}
\begin{tikzcd}[sep = large]
\prod_i \Fun(J_i, \cat C) \ar[r,"T = \prod_i \colim"] & \prod_i \cat C \\
\Fun(\prod_i J_i, \cat C) \ar[r, "W = \colim"'] \ar[u, "U"] & \cat C \ar[u, "V = \text{diag}"']
\end{tikzcd}
\end{equation}
where $U$ is the left Kan extension. Then (AB6) means that the dual Beck--Chevalley condition holds, i.e., $W\circ U^R \isoto V^R\circ T$.

Observe that $\eqref{Efimov:AB6} \simeq \cat C\otimes (\text{same square for $\Sp$})$. Using that $\cat C\otimes \blank$ is a 2-functor on $\Pr^{\mathrm{acc}}_{\mathrm{st}}$ and using (AB6) for $\Sp$, then it follows that (AB6) holds for $\cat C$.
\end{proof}

Suppose we have a short exact sequence
\[
0\to \Ind(\cat A) \xrightarrow{F} \cat C \xrightarrow{G} \Ind(\cat B) \to 0
\]
with $F$ and $G$ strongly continuous.

\textbf{Question:} Is it true that $\cat C$ is dualizable?

Consider the sequence of adjunctions $G\dashv G^R \dashv G^{RR}$. We have
\[
\cat C = \langle G^R(\Ind(\cat B)), F(\Ind(\cat A))\rangle \simeq \Ind(B) \oplus_{\Phi} \Ind(\cat A),
\]
where $\Phi \coloneqq G^{RR}\circ F$ is the gluing datum, imposing that $\Hom_{\cat C}(x,y) = \Hom_{\Ind(\cat B)}(x, \Phi(y))$ for any $x\in G^R(\Ind(\cat B))$ and $y\in F(\Ind(\cat A))$.

Observe (almost tautologically) that 
\begin{align*}
\Fun^{\mathrm{acc}}(\Ind(\cat A), \Ind(\cat B)) &= \Fun(\cat B, \Pro(\Ind(\cat A)))^{\op} \\
\Phi &\mapsto \Psi.
\end{align*}

\begin{prop}\label{Efimov:Tate}
With the above notation, the following are equivalent:
\begin{enumerate}[(1)]
\item $\cat C$ is dualizable.
\item $\cat C$ is compactly generated.
\item $\Image(\Psi) \subseteq \Tate(\cat A)$, the idempotent-complete stable subcategory of $\Pro(\Ind(\cat A))$ generated by $\Pro(\cat A)$ and $\Ind(\cat A)$.
\end{enumerate}
\end{prop}

\begin{cor}
We have
\begin{align*}
\Ext^1(\cat B, \cat A) &= \{0\to \cat A\to \cat D\to \cat B\to 0\} \\
&\simeq \Fun(\cat B, \Tate(\cat A))^{\simeq}.
\end{align*}
\end{cor}

\begin{proof}[Proof of \cref{Efimov:Tate}]
We have $\cat A \subseteq \cat C^\omega$. So if $\cat C$ is dualizable, then
\[
(\cat C/\Ind(\cat A))^\omega = (\cat C^\omega/\cat A)^{\mathrm{idem}},
\]
hence for all $x\in (\cat C/\Ind(\cat A))^{\omega}$, there exists $y\in \cat C^{\omega}$ such that $y\mapsto x\oplus x[1]$. We deduce (1)$\iff$(2).

For the equivalence (2)$\iff$(3) we use that for $x\in \cat B$ the following are equivalent:
\begin{enumerate}[(i)]
\item There exists a lift $y\in \cat C^\omega$ of $x\oplus x[1]$.
\item There is a fiber/cofiber sequence $U\to \Psi(x) \oplus \Psi(x)[1] \to V$ such that $U\in \Pro(\cat A)$ and $V\in \Ind(\cat A)$.

Equivalently, $\Psi(x) \oplus \Psi(x)[1] \in \Tate_{\mathrm{el}}(\cat A)$. By Thomason--Trobaugh's theorem, this is equivalent to $\Psi(x) \in \Tate(\cat A)$.
\end{enumerate}
\end{proof}


\begin{ex}
Consider the sequence 
\[
0\to \Perf_{\textnormal{$p$-tors}}(\ZZ) \to \Perf(\ZZ) \to \Perf(\ZZ[p^{-1}]) \to 0.
\]
The corresponding $\ZZ$-linear functor $\Perf(\ZZ[p^{-1}]) \to \Tate(\Perf_{\textnormal{$p$-tors}}(\ZZ))$ sends $\ZZ[p^{-1}] \mapsto \QQ_p$.
\end{ex}

\begin{prop}
Let $k$ be a field and let $\cat C = \{(V,W\in D(k); \varphi\colon \bigoplus_{\NN} V \to \bigoplus_{\NN} W)\}$. A direct computation shows
\[
\Psi(k) = \indprojlim_{f\colon \NN\to \NN} X_f,\qquad \text{where $X_f \coloneqq \bigoplus_{n\in\NN} k^{f(n)}$,}
\]
which is not a Tate object. We need to show that $(\ol X_f)_f$ in $\Pro(\Calk(k))$ is not pro-constant. This uses that for $f\le g$ the transition map $X_g\to X_f$ is a split epimorphism. If $(\ol X_f)_f$ were pro-constant, then $(\ol X_f)_f$ would be eventually constant, which is false since $\fib(X_{f+1}\to X_f) = \bigoplus_{\NN}k$.
\end{prop}

Let $R$ be a commutative noetherian ring. Then Neeman showed
\begin{align*}
\begin{Bmatrix}
\text{localizing subcategories} \\
\text{of $D(R)$}
\end{Bmatrix} &\cong \{\text{subsets of $\Spec(R)$}\}, \\
D_S(R) = \langle \kappa(p)\,|\, p\in S\rangle &\mapsfrom S
\end{align*}

Neeman shows that $D_S(R) \to D(R)$ is strongly continuous if and only if $S$ is closed under specialization.

\begin{thm}
The following are equivalent for $S\subseteq \Spec(R)$.
\begin{enumerate}[(i)]
\item $D_S(R)$ is dualizable.
\item $S$ is convex, i.e., if $x\leadsto y\leadsto z$ with $x,z\in S$, then $y\in S$.
\item $D_S(R)$ is compactly generated.
\end{enumerate}
\end{thm}

\begin{ex}
Let $k$ be a field, take $\cat C = \langle M\,|\, M[x^{-1}]/y M[x^{-1}] = 0\rangle \subseteq D(k[x,y])$. Then $\cat C$ is not dualizable. 

Intuitively, $\cat C = \QCoh(\AA^2/\{x=0, y\neq 0\})$.
\end{ex}

\section{Talk 3: Localizing invariants of categories of sheaves}
Recall the category 
\begin{align*}
\Shv_{\ge0}(\RR;\Sp) &\simeq \Shv_{\RR\times \RR_{\ge0}}(\RR; \Sp) \\
&\simeq \Stab^{\mathrm{cont}}(\RR\cup \{\infty\}) 
= \set{F\colon \RR_{\le}^{\op} \to \Sp}{\forall a\in \RR, F(a) = \varinjlim_{b<a} F(b)}.
\end{align*}

\begin{prop}
Take any accessible localizing invariant $\Phi\colon \Cat^{\mathrm{perf}} \to \cat E$, where $\cat E$ is a stable accessible category. Then
\[
\Phi^{\mathrm{cont}}(\Shv_{\ge0}(\RR;\Sp)) = 0.
\]
\end{prop}

\begin{applications}
\begin{enumerate}[(a)]
\item $K\colon \Cat^{\mathrm{perf}} \to \Sp$ commutes with small products.
\item Computation of $F^{\mathrm{cont}}(\Shv(X;\cat C))$, where $X$ is a finite CW complex and $\cat C$ is a dualizable category.
\end{enumerate}
\end{applications}

\textbf{Step 1:} $K_0$ commutes with small products.
\begin{prop}[Heller's criterion]
If $\cat T$ is a small triangulated category, and $x,y\in \cat T$, then the following are equivalent:
\begin{enumerate}[(i)]
\item $[x] = [y]$ in $K_0(\cat T)$.
\item There exist $z,u,v\in \cat T$ and distinguished triangles 
\begin{align*}
u \to x\oplus z \to v \\
u \to y\oplus z \to v.
\end{align*}
\end{enumerate}
\end{prop}

\begin{cor}
$K_0(\prod_i \cat T_i)\isoto \prod_i K_0(\cat T_i)$.
\end{cor}

\textbf{Step 2:} We have a short exact sequence
\[
0\to \Shv_{\ge0}(\RR; \Sp) \to \Fun(\QQ_{\le}^{\op}, \Sp)\xto{F} \prod_{\QQ} \Sp \to 0,
\]
where $F$ is given by
\[
F(G)_a = \Cone\bigl(\varinjlim_{b>a} G(b) \to G(a)\bigr).
\]
Note that $F^R$ is fully faithful, i.e., $F\circ F^R = \id$. It follows that
\begin{align*}
\Shv_{\ge0}(\RR; \Sp) &\simeq F^R\Bigl(\prod_{\QQ}\Sp\Bigr)^{\perp} \simeq {}^{\perp} F^R\Bigl(\prod_{\QQ}\Sp\Bigr) = \Ker(F).
\end{align*}
Note also that the categories $\cat A\coloneqq \Fun(\QQ_{\le}^\op, \Sp)$ and $\cat B \coloneqq \prod_{\QQ}\Sp$ are compactly generated.

We need to show that $F^{\omega} \colon \cat A^\omega \to \cat B^\omega$ is a K-equivalence, i.e., there exists $G\colon \cat B^\omega \to \cat A^\omega$ such that $[F^\omega\circ G] = [\id]$ in $K_0(\Fun(\cat B^\omega, \cat B^\omega))$ and $[G\circ F^\omega] = [\id]$ in $K_0(\Fun(\cat A^\omega, \cat A^\omega))$.

Here, we take $G\colon \cat B^\omega = \bigoplus_{\QQ}\Sp^{\omega} \to \cat A^\omega$ corresponding to $(h_a)_{a\in \QQ}$, where
\[
h_a(b)= \begin{cases}
\SS, & \text{if $b\le a$}, \\
0, & \text{if $b>a$}.
\end{cases}
\]
With this definition, $F^\omega\circ G = \id$. It remains to show $[G\circ F^\omega] = [\id]$ in $K_0(\Fun(\cat A^\omega, \cat A^\omega))$.

\begin{prop}
$K_0(\Fun(\cat A^\omega, \cat A^\omega)) = \End(\bigoplus_{\QQ}\ZZ)$.
\end{prop}

\textbf{Step~3:}
\begin{prop}
Let $\cat C$ be a small idempotent-complete stable category with semiorthogonal decomposition
\[
\cat C = \langle \cat A_0, \cat A_1, \dotsc \rangle,
\]
meaning that $\Hom(\cat A_i, \cat A_j) = 0$ for $i>j$, and the $\cat A_i$ generate $\cat C$. Write $\cat B_n \coloneqq \langle \cat A_0,\dotsc,\cat A_n\rangle \subseteq \cat C$. Then we have functors $\cat B_{n+1} \to \cat B_n$ which are right adjoint to the inclusion. Consider the composite
\[
\varprojlim_n \cat B_n \to \cat B_k \xrightarrow{p_k} \cat A_k,
\]
where $p_k$ is the right adjoint to the inclusion. 

Then the functor
\[
\varprojlim_n \cat B_n \to \prod_{n\in \NN} \cat A_n
\]
is a K-equivalence.
\end{prop}
\begin{proof}[Sketch:]
Write $\cat B \coloneqq \varprojlim_n \cat B_n$ and write $\pi_n\colon \cat B\to \cat A_n$. Consider the inclusions $\iota_n\colon \cat A_n\to \cat B$, given by compatible functors $\cat A_n\to \cat B_n \injto \cat B_k$. Write $\iota\colon \prod_n \cat A_n \to \cat B$. 

Then $\pi\circ\iota = \id$ of $\prod_n\cat A_n$.

\begin{claim}
$[\iota\circ\pi] = [\id]$ in $K_0(\Fun(\cat B, \cat B))$.
\end{claim}
Consider the functor 
\[
\psi_n\colon \cat B\to \cat B_{n-1}^{\perp} \to \cat B,
\]
where each $\cat B_n \injto \cat B$ (so that the right orthogonal makes sense) and we put $\cat B_{-1} = 0$. Now observe: consider the exact sequence
\[
\bigoplus_{n\ge1} \psi_n \to \bigoplus_{n\ge0} \psi_n \epito \iota\circ\pi,
\]
where the first map is induced by the maps $\psi_{n+1} \to \psi_n$. We compute
\[
[\id]  = [\psi_0] = \Bigl[\bigoplus_{n\ge0} \psi_n\Bigr] - \Bigl[\bigoplus_{n\ge1}\psi_n\Bigr] = [\iota\circ\pi].
\]
\end{proof}

\begin{cor}
Let $\cat B_0\from \cat B_1 \from \cat B_2 \from \dotsb$ be an inverse system in $\Cat^{\mathrm{perf}}$ such that $\cat B_{n+1} \to \cat B_n$ has a fully faithful right adjoint. Then 
\[
K_0(\varprojlim_n \cat B_n) = \varprojlim_n K_0(\cat B_n).
\]
\end{cor}
\begin{proof}
Apply the above to $\varinjlim_n\cat B_n$ with respect to the right adjoints. Denote $\cat A_n = \Ker(\cat B_{n+1} \to \cat B)$. Then
\[
K_0\bigl(\varprojlim_n \cat B_n\bigr) \simeq K_0\Bigl(\prod_n\cat A_n\Bigr) = \prod_n K_0(\cat A_n) = \varprojlim_n K_0(\cat B_n).
\]
\end{proof}

\begin{cor}
$K_0(\Fun(\cat A^\omega, \cat A^\omega)) = \End\Bigl(\bigoplus_{\QQ}\ZZ\Bigr)$.
\end{cor}
\begin{proof}
Choose a bijection $\NN \isoto \QQ$, $n\mapsto a_n$. Let $\cat C_n \subseteq \cat A^\omega$ be a stable subcategory generated by the representable presheaves $h_{a_0},\dotsc,h_{a_n}$. Then
\begin{align*}
K_0(\Fun(\cat A^\omega, \cat A^\omega)) &= \varprojlim_n K_0(\Fun(\cat C_n, \cat A^\omega)) \simeq \varprojlim_n K_0(\Fun([n], \cat A^\omega)) \\
&= \varprojlim_n \prod_{i=0}^n K_0(\cat A^\omega) = \End\Bigl(\bigoplus_{\QQ}\ZZ \Bigr),
\end{align*}
since we know that
\[
K_0(\cat A^\omega) = \bigoplus_{\QQ} K_0(\Sp^\omega) = \bigoplus_{\QQ}\ZZ.
\]
\end{proof}

Recall the functor $G\colon \cat B^\omega \to \cat A^\omega$ from the above. Then $(G\circ F^\omega)(h_a) = h_a$, and this finally implies $[G\circ F^\omega] = [\id]$.

\begin{thm}
\begin{enumerate}[(1)]
\item Let $F,G\colon \Cat^{\mathrm{perf}} \to \cat E$ be accessible localizing invariants, let $\varphi\colon F\to G$ be map, and suppose that $\cat E$ has a non-degenerate t-structure. Suppose moreover that the induced map $\pi_0\varphi\colon \pi_0F \to \pi_0G$ on connected components is an isomorphism.

Then $\varphi\colon F\isoto G$ is an isomorphism.

\item $K\colon \Cat^{\mathrm{perf}} \to \Sp$ commutes with small products.
\end{enumerate}
\end{thm}
\begin{proof}
For the proof that (1)$\implies$(2) we need to show: for any set $I$, the map
\[
K\Bigl(\prod_I \cat C\Bigr) \xto{\varphi} \prod_IK(\cat C)
\]
is an isomorphism. Observe that the source and target are localizing invariants in $\cat C$ and $\pi_0\varphi$ is an isomorphism.

It remains to prove (1). The fact that $\pi_0\varphi$ is an isomorphism implies that $\pi_n\varphi$ is an isomorphism for $n\le 0$. Now, consider the resolution
\[
0\to \cat C \to \Ind(\cat C^{\omega_1}) \to \Calk_{\omega_1}(\cat C) \to 0
\]
and proceed by induction.
\end{proof}

We also deduce from the proof that for all dualizable categories $\cat D$, $\pi_n\varphi^{\mathrm{cont}}_{\cat D}$ is an isomorphism for $n\le -1$. Then we use
\[
0\to \Shv_{>0}(\RR;\Sp) \to \Shv_{\ge0}(\RR;\Sp) \xrightarrow{\Gamma_c(\blank)[1]} \Sp \to 0.
\]
An inductive argument shows that $\pi_n\varphi$ is an isomorphism for all $n\in\ZZ$.

\begin{cor}
For any accessible localizing invariant $F\colon \Cat^{\mathrm{perf}} \to \cat E$ we have
\[
F^{\mathrm{cont}}\bigl(\Shv(\RR\cup\{\infty\}; \Sp)\bigr) = 0.
\]
\end{cor}
\begin{proof}
Observe the exact sequence
\[
0\to \Shv_{\ge0}(\RR;\Sp) \xto{\ol\varphi_\gamma^*} \Shv(\RR\cup \{\infty\}; \Sp) \xto{\alpha} \Shv_{\le 0}(\RR; \Sp),
\]
where $\alpha$ is the left adjoint to $j_!\varphi_{-\gamma}^*$: Let $\gamma = \RR_{\le 0}$ and write $\RR_{\gamma}$ for $\RR$ equipped with the $\gamma$-topology ($U \subseteq \RR_\gamma$ is open if $U + \gamma = U$ and $U\subseteq \RR$ is open). Similarly, define $\RR_{-\gamma}$.

Then $\varphi_\gamma \colon \RR\to \RR_{\gamma}$, $\varphi_{-\gamma} \colon \RR \to \RR_{-\gamma}$ and $\varphi_\gamma\colon \RR\cup\{\infty\} \to \ol{\RR}_\gamma$.
\end{proof}

\begin{thm}
Let $X$ be a finite CW complex and let $\cat C$ be a dualizable category. Let $F\colon \Cat^{\mathrm{perf}} \to \cat E$ be an accessible localizing invariant.

Then $F^{\mathrm{cont}}(\Shv(X;\cat C)) = F^{\mathrm{cont}}(\cat C)^X$, where the right hand side is the $X$-$\infty$-groupoid.
\end{thm}

Observe that for all finite CW complexes $Y$ we have an isomorphism
\[
F^{\mathrm{cont}}(\Shv(Y\times [0,1]; \cat C)) \isoto F^{\mathrm{cont}}(\Shv(Y; \cat C)).
\]
We get a functor $G\colon (S^{\mathrm{fin}})^\op \to \cat E$ such that $G(X) = F^{\mathrm{cont}}(\Shv(X;\cat C))$ for any finite CW complex $X$. We know $G(*) = F^{\mathrm{cont}}(\cat C)$ and $G(\emptyset) = 0$.

We need to show that $G$ commutes with pullbacks. Consider a cellular embedding $X\injto Y$ of finite CW complexes, and let $X\to Z$ be some continuous map of finite CW complexes. Then we have a commutative diagram
\[
\begin{tikzcd}
\Shv(Y\sqcup_XZ; \Sp) \ar[r]\ar[d] & \Shv(Y;\Sp) \ar[d] \\
\Shv(Z; \Sp) \ar[r] & \Shv(X;\Sp)
\end{tikzcd}
\]
which is a pullback square both in $\Pr^L_{\mathrm{st}}$ and in $\Cat^{\mathrm{dual}}_{\mathrm{st}}$ (because the right vertical map is a quotient functor).

We deduce that $F^{\mathrm{cont}}$ commutes with such pullbacks:

\begin{lem}
Let
\[
\begin{tikzcd}
\cat A \ar[d] \ar[r] & \cat B \ar[d] \\
\cat C \ar[r] & \cat D
\end{tikzcd}
\]
be a pullback square in $\Pr^L_{\mathrm{st}}$. Assume that $\cat A, \cat B, \cat C, \cat D$ are dualizable, all functors are strongly continuous and the right vertical map is a localization. 

Then the left vertical map is a localization, $\cat A = \cat B\times_{\cat D}^{\mathrm{dual}} \cat C$ and
\[
F^{\mathrm{cont}}(\cat A) = F^{\mathrm{cont}}(\cat B)\times_{F^{\mathrm{cont}}(\cat D)} F^{\mathrm{cont}}(\cat C).
\]
\end{lem}


\end{document}











